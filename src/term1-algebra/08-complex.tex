$K$ --- поле, $K[x]$ --- евклидово кольцо. Тогда $f, g \in K[x], h \in K[x]$.  $f \equiv g \pmod{h}$, если  $f - g \divby h$.

 \begin{statement}
    Это отношение эквивалентности. $f_1 \equiv g_1, f_2 \equiv g_2 \pmod h$, тогда $f_1 + f_2 \equiv g_1 + g_2, f_1f_2\equiv g_1g_2 \pmod h$.
\end{statement}
\begin{proof}
    Как в целых.
\end{proof}

Из этого следует, что $\exists$ фактормножество  $K[x]/ \equiv_h$ и на это множество переносятся  $+$ и  $\cdot$: получаем ассоциативное коммутативное кольцо с 1.  $K[x] / (h)$ --- кольцо вычетов по модулю $h$. Заметим, что  $\forall f \in K[x]\  \exists! \ r \in K[x]\!: f \equiv r \pmod h$ и  $\deg r < \deg h$ по теореме о делении с остатком. Причем $\deg r < \deg h$.
 \begin{example}
     $h = x - a$.  $\forall f\!: \overline{f} = \overline{c}$,  $c=$const.  $K[x] / (x - a) \cong K$.
\end{example}
 \begin{example}
     $h = x^2 - 1$,  $\forall f\!: \overline{f} = \overline{ax + b}$

     $K[x] / (x^2-1) \cong K[x] / (x-1) \times K[x] / (x+1)$
 \end{example}

 \begin{example}
     $h = x^2 + 1$,  $K = \R$.  $K[x] / (x^2+1) = \CC$ --- поле комплексных чисел.  $\CC = \{ \overline{ax + b} \mid a, b \in \R\}$.  $\overline{ax + b} = \overline{a} \cdot \overline{x} + \overline{b}$  $\overline{x} \coloneqq i$.  $i^2 = \overline{x}^2 = \overline{x^2+1} + \overline{-1} = -1$
 \end{example}

 Итоги (на самом деле не итоги, т.к. мы это докажем шагом позже, но хз): 1: $\CC$ --- поле, 2: $\{ \overline{a} | a \in \R \}$ --- подполе, изоморфное $\R$

 Кек: $\overline{a + bx} \cdot \overline{a - bx} = \overline{a^2 - b^2(-1)} = \overline{a^2+b^2}$

 Другой кек (пруф): $\overline{a + bx} \neq \overline{0} \to \overline{a + bx} \cdot \frac1{\overline{\frac{a}{a^2 + b^2} - \frac{b}{a^2 + b^2}x}} = 1$, т.е. $\overline{a + bx}$ --- обратим.

 \begin{definition}
     
     $z = a + bi \in \CC$. Число $a - bi$ называется сопряженным к $z$ и обозначается $\overline{z}$. 
 \end{definition}
 \begin{definition}
     $a = \Re(z), b = \Im(z)$, $\Re$ --- вещественная часть,  $\Im$ --- мнимая. 
 \end{definition}

 Явные формулы для сложения: $(a+bi) + (c+di) = (a+c)+(b+d)i$, для умножения:  $(a + bi)(c + di) = (ac - bd) + (ad + bc)i$.

 Модуль комплексного числа определим как $|z| = \sqrt{a^2 + b^2}$. Тогда


 $z + \overline{z} = 2\Re(z)$,  $z - \overline{z} = 2 \Im(z) \cdot i$.  $z \cdot \overline{z} = a^2 + b^2 = |z|^2$.

 $\frac{1}{z} = \frac{\overline{z}}{|z|^2}$ --- формула для $z^{-1}$ ($z \neq 0 \iff |z| \neq 0$)  

\Subsection{Геометрический смысл комплексных чисел}
Есть биекция $I \to R^2$, то есть  $a+bi \leadsto (a, b)$, то  есть комплексному числу соответствует точка на плоскости (радиус-вектор). Причем это изоморфизм групп по сложению, что следует из правила сложения комплексных чисел. 

С геометрической точки зрения сложение двух комплексных чисел означает сложение двух радиус-векторов.

Немножко сократим область рассматриваемых чисел: $T \coloneqq \{ z \in CC \mid |z| = 1\}$. Тогда получаем, что  $\forall z \in T\ z=a+bi \Rightarrow a^2+b^2=1$. 

Тогда воспользуемся медицинским фактом:  $\forall a, b\!: a^2 + b^2=1 \iff \exists! \alpha\!: a = \cos \alpha, b = \sin \alpha$. Тогда будем такие $z$ записывать как $z_\alpha$.

Тогда посмотрим на умножение двух таких чисел: \begin{align*}
    z_\alpha\cdot z_\beta &= (\cos \alpha + i\sin \alpha)(\cos \beta + i \sin \beta) =\\
                          &= (\cos \alpha \cos \beta - \sin \alpha \sin \beta) + i(\cos \alpha \sin \beta + \cos \beta \sin \alpha) = \\
                          &= \cos(\alpha + \beta) + i\sin(\alpha + \beta) = z_{\alpha+\beta}
\end{align*}

То есть $z_\alpha \cdot z_\beta = z_{\alpha+\beta}$,  $\alpha$ --- называется аргументом  $z_{\alpha}$. 
 \begin{remark}
     $T$ --- группа по умножению:  $|1| = 1, |z_1|=|z_2|=1 \Rightarrow |z_1z_2| = 1, |z_{1}^{-1}|=1$
\end{remark}
Рассмотрим $(\R, +) a \equiv \pmod 2\pi$, если  $a-b=2\pi k, k \in \Z$. Это отношение эквивалентности (упражнение). Тогда $(\R / \equiv_\pi, +)$ --- группа углов.

Тогда  $f: (\R / \equiv, +) \to (T, \cdot)$ (причем $I \to \cos \alpha + i \sin \alpha$) --- изоморфизм. 

Вернемся теперь обратно к $z \in \CC, z \neq 0$.  $z = |z| \cdot \frac{z}{|z|}$, причем заметим, что $|\frac{z}{|z|}| = 1$, а значит $\frac{z}{|z|} = z_{\alpha}$, откуда получаем:
\begin{definition}
    $z = |z|\cdot z_\alpha = r\cdot(\cos \alpha + i \sin \alpha), r = |z|$ --- тригонометрическая форма $z$.
\end{definition}

Причем заметим, что $\forall z_1, z_2 \in \CC \setminus \{0\}\!: z_1 \cdot z_2 = (r_1 z_{\alpha_1}) \cdot (r_2 \cdot z_{\alpha_2}) = r_1r_2 \cdot z_{\alpha_1 + \alpha_2}$

Заметим, что тригонометрическая форма числа единственна, так как $z = r z_\alpha, r \in \R_+$. Тогда  $|z| = |r| \cdot |z_\alpha| = r$, то есть  $r = |z|$. Тогда из $\alpha$ --- аргумент  $\frac{z}{|z|}$ следует, что $z = r \cdot z_\alpha$ единственна ($r \in \R_+, \alpha \in \R / 2\pi$). 

Значит существует биекция между $\CC^*$ и $\R_+ \times \R / 2\pi$: $z \leadsto (r, \alpha)$  $\CC^* = \CC \setminus \{0\}$. 

\begin{remark}
    Формула умножения говорит, что такая биекция --- изоморфизм групп $(\CC^*, \cdot)$ и $(\R_+, \cdot) \times (\R / 2\pi, +)$
\end{remark}
\Subsection{О геометрических преобразований плоскости}
\begin{definition}
    Пусть  $f\!: \R^2 \to \R^2$ --- биекция.
    \begin{enumerate}
        \item $f$ называется движением, если $\forall A, B \in \R^2\!: |f(A)f(B)| = |AB|$
        \item $f$ --- преобразование подобия, если  $\forall A, B, C, D \in \R^2\!: A \neq B, C \neq D \Rightarrow \frac{|f(C)f(d)|}{|f(A)f(B)|} = \frac{|CD|}{|AB|}$
        \item $f$ называется аффинным преобразованием, если условие выше верно для случаев  $AB\parallel CD$. (это $\iff \forall$ прямая  $l$  $f(l)$ --- прямая).
    \end{enumerate}
\end{definition}
\begin{statement}
    Любое преобразование подобия --- композиция гомотетии и движения.
\end{statement}
\begin{proof}
    Возьмем $A \neq B$,  $f$ --- преобразование подобия.  $|f(A)\cdot f(B)| = k|AB| \Rightarrow \forall C, D\!: C \neq D \Rightarrow |f(C)f(D)| = k|CD|$.

    Поэтому  $h \circ f$, где  $h$ --- гомотетия с коэффициентом  $\frac{1}{k}$ --- движение ($A, B \in \R^2$): \[
        |h\circ f(A) h \circ f(B)| = \frac{1}{k}|f(A)f(B)| = \frac{1}{k} \cdot k |AB| = |AB|
    .\] 
    А значит, $h \circ f = g$,  $h^{-1}$ --- гомотетия с коэффициентом  $k$, а значит  $f = h^{-1} \circ g$.
\end{proof}
\begin{theorem}[Теорема Шаля]
    Любое движение плоскости --- параллельный перенос на вектор $\overrightarrow{x}$, поворот вокруг точки $A$ на угол  $\alpha$ или скользящая симметрия относительно прямой на расстоянии $l$.
\end{theorem}
\begin{proof}
    Довольно школьная теорема.
\end{proof}
\begin{theorem}
    Любое преобразование подобия записывается в $\CC$ с помощью  $+, \cdot, \overline{z}$.
\end{theorem}
\begin{proof}
    \slashn
    \begin{enumerate}
        \item $f$ --- преобразование подобия  $\Rightarrow$  $f = h \circ g$,  $g$ --- движение,  $h$ --- гомотетия с фиксированным центром  $\Rightarrow$ достаточно проверить для  $h$ и  $g$. 

        \item $h$ --- с центром в  $O$. $h(x) = k \cdot x, k \in \R, x \in \CC$.
        \item $g$ --- параллельный перенос на  $\overrightarrow{x} \iff z_x \in \CC$.  $g(z) = z + z_x$.
        \item  $g$ --- поворот на  $\alpha$ вокруг  $O$. Заметим, что  $\arg(z) = \beta \to \arg(g(z)) = \alpha + \beta$ и  $|g(z)| = |z|$, то есть  $g(z) = z \cdot z_\alpha$.

            Для произвольной точки: надо сначала сдвинуть в начало координат, затем повернуть, а потом восстановить центр обратно.
        \item Симметрия относительно  $y=0$ ---  $g(z) = \overline{z}$.

            Скользящая симметрия --- навернуть параллельный сдвиг.

            Симметрия относительно другой прямой --- сдвиг + поворот.
    \end{enumerate}
\end{proof}
\begin{consequence}
    Композиция поворотных гомотетий --- поворотная гомотетия или параллельный перенос.
\end{consequence}
\begin{proof}[План доказательства]
    \slashn
    \begin{enumerate}
        \item Любая поворотная гомотетия задается линейной функцией $f(z) = az + b$ (смотри теорему).
        \item любая $f(z) = az + b$ --- это либо параллельный перенос ($a = 1$, поворотная гомотетия  $a \neq 1$).
    \end{enumerate}
\end{proof}
\begin{theorem}[Формула Муавра]
    $z = r(\cos \varphi + i \sin \varphi) \Rightarrow z^n = r^n(\cos(n\varphi) + i\sin(n\varphi))\ \forall n \in \Z$.
\end{theorem}
\begin{proof}
    \slashn
    \begin{enumerate}
        \item $n \in \N$ индукция по  $n$,
        \item  $n = 0$ очевидно,
        \item  $n < 0$ следует из случая  $n > 0$ и  $z^{-1} = \frac{1}{r}(\cos(-\varphi) + i\sin(-\varphi))$
    \end{enumerate}
\end{proof}
\slashn
Применения:
\begin{enumerate}
    \item $\cos(n\alpha) + i\sin(n\alpha) = (\cos \alpha + i \sin \alpha)^n = \sum_{k=0}^n \binom{n}{k} \cos^k \alpha \cdot (i\sin\alpha)^{n-k}$. Дальше следим за четностью  $n-k$. 

        Далее приравниваниваем  $\Re$ и  $\Im$  у левой и правой части. Получаем  $\cos(n\alpha)$ = $\sum_{j=0}^{\left[\frac{n}{2}\right]} \binom{n}{2j} \cos^{n-2j}(\alpha)(-1)^j\sin^{2j}(\alpha)$, $\sin(n\alpha)$ --- аналогично. 

\end{enumerate}
\begin{definition}
    $\cos(n\alpha) = T_n(\cos \alpha)$, где $T_n$ называется многочленом Чебышева.
\end{definition}
\Subsection{Извлечение корня}
$z_0 \in \CC, z_0 \neq 0$. Решим уравнение  $z^n=z_0$.  $z_0 = r_0(\cos \varphi_0 + i \sin(\varphi_0))$, $z = r(\cos \varphi + i \sin \varphi)$.

Тогда  $z^n = z_0 \iff r^n(\cos(n \varphi) + i\sin(n \varphi)) = r)(\cos \varphi_0 + i \sin \varphi_0)$.

Откуда получаем, что  $r = \sqrt[n]{r_0}$, а  $\varphi_k = \frac{\varphi_0}{n} + \frac{2\pi k}{n}$, $k \in \Z$. 
 \begin{theorem}
     Любое $z_0 \in \CC^*$ имеет ровно  $n$ корней $n$-ой степени.  
\end{theorem}
\begin{proof}
     \[
         z = \sqrt[n]{r_0}(\cos(\frac{\varphi_0}{n} + \frac{2\pi k}{n}) + i \sin (\frac{\varphi_0}{n}  + \frac{2\pi k}{n})), k = 0,1,2,\ldots,n-1
     .\] 
\end{proof}
\slashn
Заметим, что $x_0 \in \sqrt[n]{z} \Rightarrow x \in \sqrt[n]{z} \iff x^n = x_0^n \iff \frac{x^n}{x_0^n} = 1 \iff \left(\frac{x}{x_0}\right)^n = 1 \iff \frac{x}{x_0} \in \sqrt[n]{1} \iff x = x_0 \varepsilon, \varepsilon \in \sqrt[n]{1}$.
\Subsection{Корни из 1}.
Заметим что формула для корней из 1: \[
    \varepsilon_k = \cos\left(\frac{2\pi k}{n}\right) + i \cdot \sin\left(\frac{2\pi k}{n}\right),\ k=01,\ldots,n-1
.\] 

Причем заметим, что корни из 1 образуют правильный $n$-угольник. Это легко заметить, если расположить на окружности.

Причем заметим, что если $\varepsilon = \varepsilon_1$, тогда $\varepsilon_k = \varepsilon^k$ по формуле Муавра.

\begin{theorem}
    \slashn
    \begin{enumerate}
        \item $R$ --- кольцо.  $M_1(R) = \{ a \in R \mid a^n = 1\}$ --- группа.
        \item  $R = \CC \Rightarrow M_1(\CC)$ --- циклическая.
        \item $M_n(\CC) = \langle \varepsilon^k \rangle \iff (k, n) = 1$.
    \end{enumerate}
\end{theorem}
\begin{proof}
    \slashn
    \begin{enumerate}
        \item Очевидно следует из определения. Упражнение: проверить замкнутость.
        \item Очевидно из доказанного выше.
        \item Заметим, что $M_n(\CC) \cong (\Z / n\Z, +)$, а $\varepsilon^k \gets \overline{k}$

            Тогда $\Z / n \Z = \langle k \rangle \iff k$ --- образующий  $(k, n) = 1$.
    \end{enumerate}
\end{proof}
\begin{definition}
    Такие корни ($z \in M_n(\CC)$, такие что  $\langle z \rangle = M_n(\CC)$) называются первообразными корнями степени $n$.

    То есть $z$ --- первообразный корень степени  $n \iff \ord z = n$ (в группе $(\CC, \cdot)$)
\end{definition}
\begin{lemma}
    \[
        \sum_{a \in M_1} a^k = \begin{cases} 0, & \text{ если } k \centernot \divby n \\
        n, & \text{ если } k \divby n \end{cases}
    .\] 
\end{lemma}
\begin{proof}
    \slashn
    \begin{enumerate}
        \item $a^k = 1\ \forall a \leadsto 1 \underbrace{1+\ldots+1}_{n} = n$
        \item $k \centernot \divby n$  $\varepsilon$ --- первообразный корень,  $\varepsilon^k \neq 1$.  $\sum a^k = a + \varepsilon^k + (\varepsilon^2)^k + \ldots + (\varepsilon^{n-1})^k = \sum_{l=0}^{n - 1} \varepsilon^{lk} = \frac{1 - \varepsilon^{nk}}{1 - \varepsilon^k} = \frac{0}{1 - \varepsilon^k} = 0$
    \end{enumerate}
\end{proof}
\Subsection{Дискретное преобразование Фурье}
$f \in \CC[x]$,  $\deg f < n$,  $f = \displaystyle \sum_{k=0}^{n-1} a_kx^k$.

Тогда  $f \to (a_0, a_1,\ldots, a_{n-1})$. А ДПФ делает $(a_0, a_1, \ldots, a_{n-1}) \mapsto (b_0, b_1, \ldots, b_{n-1})$

$b_i = f(\varepsilon_i) = f(varepsilon^i)$.

\begin{theorem}[Обратное преобразование Фурье]
    \[
        a_i = \frac{\sum_{j=0}^{n-1}(e^{-i})^j b_j}{n}
    .\] 
\end{theorem}
\begin{proof}
    $\forall j = 0, 1, \ldots, n - 1$ $f(\varepsilon^j) = b_j$

    \[\sum_{i = 0}^{n-1} b_i \varepsilon^{j(i-i_0)} = b_j.\] 
    Фиксируем $i_0$, делим на  $\varepsilon^{ji_0}$:
    \[
    \sum_{i=0}^{n-1} a_i \varepsilon^{j(i-i_0)} = b_j \varepsilon^{ji_0}
    .\] 

    Сложим:
    \[
    \sum b_j \varepsilon^{ji_0} = \ldots = \sum_{i=0}^{n-1}a_i\left(\sum_{a \in M_n(1)} a^k = \sum_{i=0}^{n-1} a^k n \right)
    .\] 
\end{proof}
\Subsection{Быстрое умножение многочленов}

$f, g \in \CC[x]$ хотим  $f\cdot g$. По правилу свертки -  $\mathcal{O}(n^2)$ умножений.

Можно так  $\deg f, \deg g < n$. Тогда применяем ДПФ к  $f, g$. Получили точки. В них значения перемножили, вернули обратно после этого при помощи обратного преобразования Фурье.

\Subsection{TODO: название}

\begin{definition}
    $K$ --- поле.  $K$ называется алгебраически замкнутым, если  $\forall \in K[x] \deg f > 0 \Rightarrow f$ имеет корень в  $K$.    
\end{definition}
\begin{exerc}
    $\Q, \R, \Z / p \Z$ --- не алгебраически замкнутые.
\end{exerc}
\begin{theorem}
    Для любого поля $K$  $\exists \overline{K}: K \subset \overline{K}$ и $\overline{K}$ --- алгебраически замкнуты.
\end{theorem}
\begin{proof}[Набросок докзательства]
    Пусть $k$ не алгебраически замкнуто.  $f \in K[x]$ не имеет корней. Не умаляя общности  $f$ --- неразложимый. Рассмотрим  $K[x] / (f)$ --- поле, если  $f$ неразличимый.

    Тогда  $f_1 \equiv f_2 \pmod f$, если $f_1 - f_2 \divby f$. Тогда получаем $K \to K[x] / f$ --- инъективный гомоморфизм ($a \leadsto \overline{a}$). 

    Тогда $f(\overline{x}) = \overline{f(x)} = 0$.  $\overline{x}$ --- корень  $f$ в  $K[x] / f$

    То есть в  $K[x] / f$ f --- имеет корень. И так далее (трансфинитная индукция) \dots
\end{proof}
\begin{theorem}[Основная теорема алгебры]
    $\CC$ --- алгебраически замкнуто.
\end{theorem}
\begin{proof}
    Доказательства не будет:) Будут на-/вбросы.
\end{proof}
\begin{proof}[Наброс 1 (Топология)]
    Рассмотрим движение произвольной точки по окружности. Кукареку, никому не интересно. Либо сделайте PR))))
\end{proof}
\begin{consequence}
    $K$ --- алгебраически замкнуто  $\Rightarrow \forall f \in K[x]$  $f = a_0(x-1)(x-x_2)\ldots(x-x_n)$
\end{consequence}
\begin{proof}
    Индукция по $\deg f$
     
    $n \to n+1$.  $\deg f = n+1$,  $K$ --- алгебраически замкнуто  $\Rightarrow$ f --- имеет корень  $\Rightarrow f = (x-z_0)\widetilde{f}$, причем $\deg \widetilde{f} = n$. Переход индукции. Получаем $f = a_0(x-x_1)\ldots(x-x_n)$
\end{proof}
\begin{lemma}
    $f \in \R[x], z \in \CC$  $f(z) = 0 \Rightarrow f(\overline{z}) = a$.
\end{lemma}
\begin{proof}
    $f(x) = \sum a_kx^k$.  $f(\overline{z}) = \sum a_k(\overline{z})^k = \sum a_k\overline{z^k} = \sum \overline{a_kz^k} = \overline{0} = 0$.
\end{proof}
\begin{theorem}
    $f \in \R[x], \exists a \in \R, x_1, x_2, \ldots, x_k \in \R$.

    $p_1, \ldots, p_l, q_1, q_2,\ldots, q_l \in \R$, такие что $f = a(x-x_1)(x-x_2)\ldots(x-k) \cdot (x^2+p_1x + q_1) \ldots (x^2+p_l x + q_l)$. Причем дискриминант у скобочек меньше нуля, а $k+2l = \deg f$.
\end{theorem}
\begin{proof}
    Индукция по $\deg f$. 

    $n \to n+1$:  $\deg f = n + 1$  $f \in \R[x] \subset \CC[x] \xRightarrow[\text{ОТА}]{} \exists z \in \CC\!: f(z) = 0$.

    Пусть  $z \in \R$. Тогда просто по Безу получаем  $f=(x-z)\widetilde{f}$.
    
    $z \notin \R$. По лемме $f(\overline{z}) = 0$. Тогда раз  $(x-z, x - \overline{z}) = 1$ и  $f \divby x - z, x - \overline{z}$, то  $f \divby (x-z)(x-\overline{z}) = x^2 - (z + \overline{z})x + z\overline{z}$. Тогда $f = (x^2 + px + q)\widetilde{f}$, где  $\deg f = n - 1$. По индукции все ок)
\end{proof}
\begin{example}
    Над $\CC$:  $f = \prod (x-\varepsilon_k)$.

    Над  $\R$:  $n$ --- нечетно:  $x^n - 1 = (x-1)\prod_{k=1}^{n-1}(x - \varepsilon_k) = (x-1)\prod_{k=1}^{\frac{n-1}{2}}(x-\varepsilon_k)(x-\varepsilon_{n - k}) = (x-1)(x^2 - \cos\frac{2 \pi i}{n})$

    Над $\Q$: $\displaystyle x^n - 1 = \prod_{d \mid n}(\prod_{\varepsilon \in M_n, \ord(\varepsilon) = d}(x - \varepsilon))$. Под первым сумматором: $\Phi_d(x)$
\end{example}
\begin{statement}
    \begin{enumerate}
        \item $\Phi_d \in \Q[x]$ 
        \item  $\Phi_d$ --- неразложимы в  $\Q[x]$.
    \end{enumerate}
\end{statement}
\begin{proof}
    \begin{enumerate}
        \item Индукция по $n$.  $\Phi_n(x) = \frac{x^n - 1}{\prod_{d \mid n}\Phi_d(x)} \in \Q[x]$
        \item очень сложна и непонятно. Если бы мы знали, что это такое!
    \end{enumerate}
\end{proof}
\Subsection{Гауссовы числа}
\begin{definition}
    $\Z[i] = \{a + bi \mid a, b \in \Z\}$.
\end{definition}
\begin{statement}
    $(\Z[i], +, \cdot)$ --- кольцо.
\end{statement}
\begin{theorem}
    $\Z[i]$ --- евклидово
\end{theorem}
\begin{proof}
    $\varphi(z) = |z|^2$ --- евклидова норма.

    $a+bi, c+di \in \Z[i] c^2 + d^2 \neq 0$. 

    Хотим  $a+bi=(c+di)\cdot z +r,\ z,r \in \Z[i]$. Причем $|r| < |c+di|$. Возьмем  $z_0 = \frac{a+bi}{c+di}$. $a+bi = (c+di)z_0$. Тогда  $a+bi = (c+di)z + \underbrace{(c+di)(z_0-z)}_{=r}$. 

    Теперь хочется  $|(c+di)(z_0-z)| < |c+di| \iff |z_0 - z| < 1$. Заметим, что достаточно посмотреть на ближайшие 4 числа.
\end{proof}

\begin{theorem}[Рождественская теорема Ферма]
    $p = 4k + 1$ --- простое  $\Rightarrow x, y \in \Z p = x^2+y^2$
\end{theorem}
\begin{proof}
    Рассмотрим $p$ как элемент  $\Z[i]$.  $p = p + 0 \cdot i$.  $\exists i \in \Z: x^2 + 1 \cdot p$. Так как  $a^{p-1} = 1, a^{\frac{p - 1}{2}} = -1, (a^{\frac{p-1}{4}})^2 = -1$. $a^{\frac{p-1}{4}} = ix$.

    Тогда $x^2+1 = (x-i)(x+i) \divby p$ в  $\Z[i]$. Тогда  $p$ --- составное в  $\Z[i]$. Тогда  $p = (a+bi)(c+di)$,  $bc + ad = 0$, откуда получаем  $c = a, d = -b$, тогда  $p = (a+bi)(a-bi) = a^2+b^2$. 
\end{proof}
