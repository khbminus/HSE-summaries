Теперь мы многочлены будем рассматривать как числа (а не как к функциям). Ведь их можно складывать и умножать! Причем свойства умножения и сложения удовлетворяет требованием кольца! Получили \textbf{Кольцо многочленов над кольцом $\R$}.

Еще можно рассматривать кольцо многочленов как \textbf{кольцо формальных степенных рядов}.
\begin{definition}
    Пусть $R$ --- ассоциативное коммутативное кольцо. Тогда кольцо формальных степенных рядов  $R[[x]$ -- тройка  $(R^{\Z_{\ge 0}}, +, \cdot)$.

    $+$: $(a_0, a_1, a_2, \ldots) +  (b_0, b_1, b_2, \ldots) \coloneqq (a_0 + b_0, a_1 + b_1, \ldots)$

    $\cdot$ (Правило свертки):  $(a_0, a_1, a_2,\ldots) \cdot (b_0, b_1,b_2,\ldots) = (a_0, b_0, a_0b_1+b_1b_0,\ldots)$, по факту: $(a_i)\cdot(b_i) = c_i, c_n \coloneqq  \sum_0^n a_kb_{n-k}$

    Так же можно представлять $(a_0, a_1, a_2,\ldots) \iff a_0 + a_1x + a_2x^2+\ldots$. То есть, если неформально, то правило свертки --- обычное раскрытие скобок.
\end{definition}
\begin{definition}
    $R^{\Z_{\ge 0}} = \{ f: \Z_{\ge 0} \to R\} = \{(a_0, a_1, \ldots) | a_i \in R\}$
\end{definition}

\begin{theorem}
    $R[[x]]$ --- ассоциативное, коммутативное кольцо. Причем, если  $R$ с единицей, то  $R[[x]$ --- кольцо с единицей.
\end{theorem}
\begin{proof}
    Заметим, что все аксиомы доказывается супер просто, ведь сложение у нас просто по координатам. Тогда получили очевидность коммутативности и ассоциативности $+$. В качестве нуля берется  $0 =(0, 0, 0, 0,\ldots)$. $-(a_0, a_1,a_2,\ldots) = (-a_0, -a_1,-a_2\ldots)$

    Дистрибутивность --- упражнение (из дистрибутивности $R$).
    
    Коммутативность произведения: $c_n = \sum_0^m a_k b_{n-k} = \sum a_k b_l$, где  $k, l \ge 0 \land k+l=n$. Тогда $c_n = \sum_{l=0}^n a_{n-l} b_l = \sum_{l = 0}^m b_l a_{n-l}$ --- формула свертки для $b \cdot a$.

    Если  $\exists 1_R$, то  $(1_R, 0_R, 0_R, \ldots)$ --- нейтральный относительно $\cdot$ в  $R[[x]$ (упражнение).

    Ассоциативность (упражнение на смирение духа):  $\forall f, g, h \in R[[x]] (f\cdot g) \cdot h = f \cdot (g \cdot h)$. Введем много обозначений: $f=(a_n), g=(b_n), h=(c_n), f \cdot g = (d_n), g \cdot h = (e_n), (f \cdot g) \cdot h = k_n, f \cdot (g \cdot h) = (?_n)$

    $k_n = l_n\ \forall n \in \Z_{\ge 0}$. Тогда \[
        k_n = \sum_{i=0}^n d_i c_{n-i} = \sum_{i=0}^n (\sum_{j=0}^i a_jb_{i-j}) c_{n-i}
    .\]  
    Воспользуемся дистрибутивностью: \[
        k_n = \ldots = \sum_{\mathclap{\substack{0 \le i \le n \\ 0 \le j \le i}}} a_j b_{i-j} c_{n-i}
    .\] 
    Определим $s \coloneqq i - j, t \coloneqq n - i$, тогда  \[
    k_n = \ldots = \sum_{\mathclap{\substack{j, s, t \ge 0 \\ j+s+t = n}}} a_jb_sc_t.
    .\] 
    Аналогично для $l_n$:  \[
    l_n = \ldots = \sum_{\mathclap{\substack{j, s, t \ge 0 \\ j+s+t = n}}} a_jb_sc_k.
    .\] 
\end{proof}
\begin{remark}
   $R$ --- не коммутативное кольцо. Тогда стоит различать  $ax^2, x^2a, xax$. 
\end{remark}
\begin{remark}
    Существует инъективный гомоморфизм колец $i: R \to R[[x]]$: $a \to (a, 0, 0, 0,\ldots)$. \textit{Это можно проверить}. 

    Тогда не умаляя общности считаем, что $R$ содержится в  $R[[x]]$.
\end{remark}
\begin{remark}
    Положим по определению $x \coloneqq (0, 1, 0, 0, 0, \ldots)$. 

    Тогда (упражнение на индукцию) $x^n \coloneqq (0, 0,\ldots, \overbrace{1}^n, 0, 0, \ldots)$. 

    Тогда, если $f= (a_0, a_1,a_2,\ldots,a_n, 0, 0, 0)$ ($a_i$ при  $i > n$ равно 0). 

    Тогда $f = a_0 + a_1 \cdot x + a_2 \cdot x^2 + \ldots + a_n \cdot x^n$. 
\end{remark}
\begin{remark}
    $(a_0, a_1, a_2,\ldots) \cdot \underbrace{(0, 1, 0,\ldots)}_{x} = (0, a_0, a_1, \ldots)$
\end{remark}
\begin{consequence}
    $f \divby x$.  $f = (a_i) \land a_0 = 0 \Rightarrow 1 \centernot \divby f$. 
\end{consequence}
\begin{theorem}
    $f = (a_i)$.  $f \in R[[x] \iff a_o \in R^*$. В частности: $R$ --- поле  $\Rightarrow f$ --- обратим $\iff f \centernot \divby x$. 
\end{theorem}
\begin{proof}
    \slashn
     \begin{itemize}
         \item $\Rightarrow$.  $(a_0, a_1, \ldots) \cdot (b_0, b_1, \ldots) = (1, 0, 0,\ldots)$.

             $1 = a_0b_0 \Rightarrow a_0\in R^*$.
         \item $\Leftarrow$: будем вычислять последовательность  $(b_0, b_1,\ldots)$. $a_0 \in R^*$, тогда: 

             $a = a_0b_0 \Rightarrow b_0 = a_0^{-1} = \frac{1}{a_0}$. $0 = a_0b_1 + a_1b_0 \Rightarrow \frac{-a_1b_0}{a_0}$. И так далее.

             $0 = \sum_{i=0}^n a_i b_{n-i}$.  $b_n = (-\sum_{i=1}^n a_i b_{n-i})a_0^{-1}$.  

             Построили метод построения $b$, причем все хорошо!
    \end{itemize}
\end{proof}
\begin{example}
    $f = (1, 1, 1, 1, \ldots) = 1+x+x^2+x^3+\ldots$. Тогда $\frac{1}{1+x+x^2+\ldots} = 1-x$. Тогда $1+x+x^2+x^3+\ldots = \frac{1}{1-x}$.
\end{example}
\begin{theorem}
    Подмножество в $R[[x]]$  $R[x] = \{(a_0,a_1,\ldots \mid \exists N \forall n > N\!: a_n=0\}$ --- финитные последовательности, образуют подкольцо с единицей, называемое кольцом многочленов.
\end{theorem}
\begin{proof}
    Замкнутость по $+$:  $a_n = 0$ при  $n > N_1$ и  $b_n = n> N_2$. Тогда при $n > \max(N_1, N_2) a_n+b_n = 0$.

    Замкнутость по $\cdot$:  $a_n = 0, n > N_1$ и $b_n = 0, n > N_2$. Тогда при  $n > N_1+N_2:$ $c_n = \sum_{i+j=n} a_ib_j = 0$. Так как при  $i + j = N > N_1+N_2 \Rightarrow i > N_1 \lor j > N_2$.

    $1 \in k[x]$!!!
\end{proof}
\begin{definition}
    $f \in k[x]$ степенью  $f$ называется  $\deg f = \{\max k: a_k \neq 0\}$. Причем $\deg 0 = -\infty$
\end{definition}
\begin{properties}
    \slashn
    \begin{enumerate}
        \item $\deg (f+g) \le \max(\deg f, \deg g)$. Причем $\deg f \neq deg g: \deg(f+g) = \max(deg f, \deg g)$. 
        \item $\deg(f\cdot g) \le \deg f + \deg g$, если $R$ --- область целостности, то  $\deg (fg) = \deg f + \deg g$.
    \end{enumerate}
\end{properties}
\begin{consequence}
    $R$ --- область целостности  $\Rightarrow R[x]$ --- область целостности.
\end{consequence}
\slashn
Теперь у нас $k$ --- поле.
\begin{theorem}[О делении с остатком]
    $f, g \in k[x]$ $g \neq 0$. Тогда  $\exists! q, r \in k[x]: f = g\cdot q + r, \deg r < \deg g$.
\end{theorem}
\begin{consequence}
    $R$ --- коммутативное, ассоциативное кольцо  $a \in R$. Тогда $\exists$ гомоморфизм колец  $R[x] \to R: a_0 + a_1 x + \ldots + a_n x^n \mapsto a_0 + a_1 \cdot a + \ldots + a_n a^n$ --- гомоморфизм эвалюации. 

    С другой стороны $f \in R[x]$ --- полиномиальная функция.  $F_f: R \to R$  $a \mapsto \text{ev}_a(f)$.
\end{consequence}
\begin{definition}
    $f \in R[x]$.  $a\in R$ --- корень  $f$, если  $F_f(a) = 0$.
\end{definition}
\begin{theorem}[Безу]
    $K$ --- поле.  $f \in K[x]$. $a \in K$. $f = (x-a)g + r$ --- деление с остатком.
    \begin{enumerate}
        \item $r = f(a)$.
        \item  $a = 0 \iff f \divby (x-a)$
    \end{enumerate}
\end{theorem}
\begin{proof}
    $f = (x-a) \cdot g + r$,  $\deg r < \deg (x-a) = 1 \Rightarrow \deg r = 0 \lor \deg r = -\infty \iff r = c \in K$.

    $F_f(a) = F_{x-a}(a)F_g(a) + F_r(a)$.  $f(a) = (a - a)g(a) + r \iff r = f(a)$. 
\end{proof}
\begin{consequence}
    $\deg f = n, f \in K[x], f \neq 0 \Rightarrow$ существует не более  $n$ корней  $f$ в  $K$.
\end{consequence}
\begin{proof}
    По индукции по $n$.
     \begin{itemize}
         \item База $n = 0$  $f=r \neq 0$ --- 0 корней.
         \item Переход  $n \to n+1$:

              $\deg f = n + 1$. Нет корней  $\Rightarrow 0 \le n + 1$.

              Существует $a$ --- корень.  $f = (x-a) \widetilde{f}, \deg \widetilde{f} = n$. У $\widetilde{f}$ не более  $n$ корней  $\Rightarrow$ у  $f$ не более  $n+1$ корня.

              С другой стороны $b$ --- корень  $f \Rightarrow f(b) = 0$. $(b-a) \widetilde{f}(b) = 0 \xRightarrow{k\text{ --- о. ц. }} b - a = 0 \lor \widetilde{f} = 0 \iff b = a \lor b$ --- корень $\widetilde{f}$. Таких не более $n$, а значит у  $f$ не более  $n+1$ корня.
    \end{itemize}
\end{proof}
