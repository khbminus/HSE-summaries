\Subsection{Высказывания}

Начало пропущено, но оно простое.

Высказывание --- утверждение, которое может быть ложным или истинным.

Тут был кек про Юрия Зайцева и про то, как из ложного утверждения следует всё, что угодно. 

\begin{definition}
Пропозициональная переменная --- произвольная переменная, обозначаем $x, y, z$, могут принимать значения 0 или 1
\end{definition}

\begin{definition}
Формула: 

\begin{enumerate}
	\item{Пропозициональная переменная является формулой}

	\item{Если $A$ --- формула, то $\lnot A$ --- тоже формула (отрицание)}

	\item{Если есть формулы $A, B$, то $(A \cap B), (A \cup B), (A \to B)$ --- тоже формулы}

\end{enumerate}

Минимальный класс строк, удовлетворяющий $*$ (предыдущей тройке утверждений), называется множеством формул.

\end{definition}

Теперь хотим давать значения формулам. Для переменной значение формулы есть просто значение переменной. Отрицание работает как отрицание, логические операции работают согласно таблице истинности (которая очевидна)

Для отсутствия проблем с порядком вычисления у нас есть скобки.

\begin{theorem}
	Любая формула допускает единственный разбор.
\end{theorem}
\begin{proof}
	Когда у нас есть формула, мы однозначно попадаем в один из трёх вариантов. Формально говоря --- посмотрим первый символ. Если это переменная, то мы завершаемся, если отрицание --- отрицаем и убираем отрицание, если открывающая скобка --- то в общем там можно найти операцию, в которой скобоный баланс равен единице и порвать по этой операции и уйти рекурсивно с двумя меньшими строками.
\end{proof}

\begin{definition}
	Тавтология --- формула, истинная при любых значениях переменных. Например $x \cup \lnot x$, или $(p \to q) \cup (q \to p)$
\end{definition}
\begin{definition}
	Две формулы $F_1, F_2$ эквивалентны, если $F_1$ истинно тогда и только тогда, когда $F_2$ истинно.
\end{definition}

На самом деле $F_1, F_2$ эквивалентны тогда и только тогда, когда $(F_1 \to F_2) \land (F_2 \to F_1)$ является тавтологией

\begin{definition}
	$A \leftrightarrow B = (A \to B) \land (B \to A)$ 
\end{definition}

\begin{theorem}
	Следующие формулы являются тавтологиями:
	\begin{itemize}
		\item{$(p \land q) \leftrightarrow (q \land p)$}
		\item{$((p \land q) \land r) \leftrightarrow (p \land (q \land r))$}
		\item{$  \lnot (p \land q) \leftrightarrow (\lnot p \lor \lnot q) $}
		\item{$\lnot (p \lor q) \leftrightarrow (\lnot p \land \lnot q)$}
		\item{$p \lor (p \land q) \leftrightarrow p$}
		\item{$p \land (p \lor q) \leftrightarrow p$}
		\item{$(p \to q) \leftrightarrow (\not q \to \not p)$}
		\item{$p \land (q \lor r) \leftrightarrow (p \land q) \lor (p \land r)$}
		\item{$p \lor (q \land r) \leftrightarrow (p \lor q) \land (p \lor r)$}
	\end{itemize}
\end{theorem}

\begin{theorem}
	Пускай есть $f : B^n \to B$ (из битовых строк длины $n$ в истину/ложь). Утверждение --- любая такая $f$ представляется используя $\land, \lor $ и $\lnot$, и, мало того, представляется в виде ДНФ: дизъюнктивно-нормальная формула.
\end{theorem}

\begin{definition}
	Конъюнкт --- $(x_{i1} \land x_{i2} \land \cdots \land x_{ik})$ --- набор логических И, возможно с отрицаниями.
	\end{definition}

\begin{definition}
	ДНФ --- дизъюнкция конъюнктов --- набор конъюнктов, объединённых логическим ИЛИ.
\end{definition}

\begin{proof}
	Докажем предыдущую теорему.

	Рассмотрим все наборы переменных и значения $f$ на них (рассмотрим таблицу истинности $f$). Выберем строки, в которых $f = 1$, тогда возьмём переменные оттуда и составим из них конъюнкт --- каждую переменную, если она со значением 0, то возьмём в конъюнкт её отрицание, а если со значение 1 --- возьмём саму переменную. Тогда полученный конъюнкт будет верен только и только на этом наборе переменных.

	Взяв все такие конъюнкты, на наборах переменных которых $f$ истинна, и объединив их через логическое ИЛИ, получим как раз ДНФ, истинную только на тех наборах переменных, на которых была истинна $f$. 
\end{proof}
\begin{example}
	Если кто хочет --- можете нарисовать тут таблицу истинности для формулы от трёх переменных и построить по ней описанным алгоритмом ДНФ. 
\end{example}

\begin{definition}
	КНФ --- конъюнкция дизъюнктов --- набор дизъюнктов (объединений переменных через $\lor$), объединённых через логическое И (выполняться должны все скобки).
\end{definition}

\begin{theorem}
	Для любой формулы можно построить КНФ. Строится аналогично, но выбираем строки с нулём. 
\end{theorem}

КНФ/ДНФ не единственны, и построенные нами не обязательно минимальны. 

Можно ли используя другие связки тоже выразить всё? 

Давайте выразим всё через $1, \oplus, \land$. 

\begin{definition}
	Одночлен Жегалкина --- $1, x, x \land y, x \land y \land w$
\end{definition}

\begin{definition}
	Многочлен Жегалкина: набор одночленов Жегалкина, объединённых $\oplus$
\end{definition}

\begin{theorem}
	Любая функция $f : B^n \to B$ допускает ровно одно представление в виде полинома Жегалкина. 
\end{theorem}

\begin{proof}
	Любой многочлен можно свести к многочлену Жегалкина: $x^2y \oplus xy \oplus xy = xy \oplus xy \oplus xy = xy$. Единственность --- при использовании каждого одночлена не более раза. 

	Мы знаем, что используя $\lnot, \lor, \land$ можно записать все формулы. А сами эти операции можно выразить следующим образом: $\lnot x = x \oplus 1; x \lor y = xy \oplus x \oplus y; x \land y = x \land y$.
\end{proof}
\begin{example}
	Если сильно нужен --- напишите, добавлю. Можете и сами добавить...
\end{example}
\begin{proof}[Единственность представления]
	У нас бывает $2^n$ одночленов (переменная или входит или не входит). Многочлен это набор одночленов, тогда многочленов $2^{2^n}$, но и всех функций $2^{2^n}$, и при этом каждой функции соответствует хотя-бы один многочлен, а тогда, т.к. функций столько же, сколько и многочленов, каждой функции соответствует ровно один многочлен.
\end{proof}

Пусть у нас есть произвольная тернарная функция $f(x, y, z)$ и $\land$. Можем ли мы представить любую функцию таким интересным набором? Хотим иметь критерий на этот счёт. 

Займёмся классами функций
\begin{definition}
	Сохраняющие ноль функции --- функция от всех нулей выдаёт ноль. Пример --- тождественный ноль, логическое или.

	Сохраняющие единицу функции --- функция от всех единиц выдаёт единицу. Пример --- тождественная единица, логическое или.

	Отрицание не является ни сохраняющей единицу, ни сохраняющей ноль.

	Монотонные функции --- от увеличения (замены нуля на единицу) любого из параметров значение функции не уменьшается. 

	Линейные функции --- те, для которых многочлен Жегалкина состоит только из $\oplus$ переменных и единиц, иначе говоря, в многочлене Жегалкина которых нет слагаемых степени больше единицы.

	Самодвойственные --- отрицание всех переменных влечёт отрицание значения функции. 
\end{definition}

\begin{example}
	$\land$ --- сохраняющая ноль, сохраняющая единицу, монотонная, не самодвойственная функция. 
	
	Тождественная функция является самодвойственной. 
\end{example}

\begin{theorem}[Теорема Поста]
	Набор функций является полным (можно представить любую функцию) тогда и только тогда, когда для любого класса из пяти названных существует $f_i$, не лежащее в этом классе.
\end{theorem}
\begin{proof}
	Если нет класса, в котором не лежит хотя-бы одна функция, то мы проиграли, т.к. не сможем выразить какую-либо функцию не из этого класса. Например, любая комбинация сохраняющих ноль/единицу функций тоже является сохраняющей ноль/единицу функцией, и тогда отрицание мы не выразим.

	Аналогично с монотонностью --- комбинация монотонных функций монотонна. Изменили увеличили какие-то переменные, функции, в которых они были могли только увеличиться, те функции, в которых эти функции как аргументы тоже могли только увеличиться и так далее... В общем, отрицание снова не получим. 

	Самодвойственность аналогичным образом проходит внутрь, к аргументам функций, которые если тоже функции, то отрицание тоже пройдёт внутрь и так далее, пока не проотрицаем все аргументы, что и есть самодвойственность. 

	И с линейностью так же будет, но в общем лекция закончилась, теорему доказать не успели.
\end{proof}
