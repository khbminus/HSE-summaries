\Subsection{Условия}
1 теория в неделю, 1 практика в неделю. \\
\begin{itemize}
    \item Теор. часть. \\ 
        Дедлайн: вторник, 23:59. Потом придут исправления, которые надо сдать до пятницы ($\le 23:59$). Сдача после дедлайна --- понижение коэффициента. Два типа задач:
        \begin{enumerate}
            \item Обязательные, $\Sigma =10-15$
            \item Дополнительные. <<Overprice>>
        \end{enumerate}
        Домашки сдавать обязательно в \TeX, если вы не в группе А. Олемской. Дедлайны можно переносить, если вам тяжело/заболели, то можно попросить перенести дедлайн лично для вас. Но если делать так слишком часто, то это неоч :(.
    \item Контест. 3 три задач:
        \begin{enumerate}
            \item Must Have. Если не сдал --- пиши-пропало.
            \item Обязательные. Сумма маст хэвов и обязательных -- 10-15 баллов.
            \item Дополнительные. <<Overprice>>
        \end{enumerate}
\end{itemize}

Не стоит сначала обращать внимание на мелочи. Пытайтесь вычленить основную идеи. Уже когда поймете её, стоит пытаться найти интересные случаи. 

Как понять что вы поняли алгоритм? Сесть и подумать: можете ли вы прямо сейчас сесть и написать код. Если не можете, то надо задавать вопрос. \textbf{Думайте, перед тем, как писать}.

Обучение ~-- интерактивный процесс, старайтесь включаться, если вы переходите в режим зрителя, то становится плохо.

\Subsection{Асимптотика}

Как выбрать процессор? У процессора есть несколько остальных характеристик (примеры в скобках): количество ядер (8 ядер), частота (3.3 GHz), набор инструкций, битность (32/64). 

У нас все алгоритмы однопоточные, поэтому для нас важна только частота.

\begin{tabular}{c | c}
    +, -, * / & 1 операция \\ \hline
    a[i] & 1 операция \\ \hline
    if & 1 операция \\ \hline
    f(..) & 1 операция \\ \hline
    if & 1 операция
\end{tabular}

\TODO{Схема}\\
Время $\to$ константа + асимптотика. Асимптотика, если просто, число операций, к которому стремится при увеличении количества входа. 

Есть асимптотика используемого времени и памяти. Утверждается, что $\text{Время} \ge \text{Память}$, потому что на выделение памяти тоже время (причем 1 ячейка = 1 операция). Второй момент, время довольно безгранично, а память конечна.
\begin{definition}
    $f=\mathcal{O}(g(n))\text{:   } \exists C>0: \exists N: \forall n \ge N: f(n) \le C \cdot g(n)$
\end{definition}
\begin{definition}
    $f=\Theta(g(n))\text{:  } \exists C_1>0,C_2>0: \exists N: \forall n \ge N: C_1\cdot g(n) \le f(n) \le C_2 \cdot g(n) $
\end{definition}
\begin{definition}
    $f = o(g(n))\text{:   } \forall C>0: \exists N: \forall n \ge N: f(n) \le C \cdot g(n)$
\end{definition}

\begin{property}
    $f \pm o(f) = \Theta(f)$
\end{property}
\begin{proof}
    $\exists N: \forall n \ge N: o(f) \le \frac{1}{2}f \Rightarrow \frac{1}{2}f \le f + o(f) \le \frac{3}{2}f$
\end{proof}

\begin{property}[транзитивности $\Theta$]
    $\Theta(\Theta(f)) = \Theta(f)$
\end{property}
\begin{proof}
    Внешняя и внутренняя $\Theta$ зажата константами, а значит можно сказать, что константы внутренней равны внешней. 
\end{proof}
\begin{lemma}
    $\forall P\text{:  } P(n) = \Theta(n^{deg\text{ }P})$
\end{lemma}
\begin{proof}
    \TODO{Я не успел, смотри конспекты Сережи.}
\end{proof}

\Subsection{Умножение. Карацуба}
$735 = 7 \cdot 100 + 3 \cdot 10 + 5 = 7x^2 + 3x + 5$, если $x=10$.
Пусть $n=x^2+x+)$, а $m=3x+7$. Тогда  $nm = (x^2+x+1)(3x+7)$. Дальше для умножения многочленов можно просто раскрыть скобочки. Если записать это в коде, то получим:
\begin{lstlisting}
    for (i = 0; i < n; i++)
        for (j = 0; j < m; j++)
            c[i + j] += a[i] * b[j]
//          x^{i+j}     x^i  * x^j
\end{lstlisting}
Данный код работает за $\mathcal{O}(n^2)$. Долго человечество не могло решить задачу быстрее. Но Анатолий Карацуба придумал быстрее:
\begin{enumerate}
    \item $n = m = 2^k$ 
    \item Разобьем $A(x)$ и $B(x)$ на две половины: $A_1$ и $A_2$, $B_1$ и $B_2$.\\
        $A(x) \cdot B(x) = (A_1(x)+ x^{\frac{n}{2}} \cdot A_2(x))(B_1(x) + x^{\frac{n}{2}} \cdot B_2(x)) = A_1B_1 + x^{n} \cdot A_2B_2 + x^{\frac{n}{2}} \cdot (A_1B_2+A_2B_1)$.
\end{enumerate}
\TODO{схемы}

Запишем время работы нашего алгоритма: $T(n) = 4T(\frac{n}{2}) + n$, что равно (магия пока что) $\Theta(n^2)$. Но давайте напишем алгоритм, работающий за $T(n) = 3T(\frac{n}{2})+n \approx n^{1.6}$.

Для этого заметим, что $A_1B_2A_2B_1=(A_1+A_2)(B_1+B_2) - A_1B_1 - A_2B_2$. Здесь сложение многочленов это операция соответственного суммирования коэффициентов перед степенями.
\begin{lstlisting}
Mul(A, B)
    A -> A1 A2
    B -> B1 B2
    c = Mul(A2, B2)
    d = Mul(A1, A2)
    e = Mul(A1 + A2, B1 + B2)
    ...
\end{lstlisting}
\Subsection{Мастер-Теорема}
\begin{theorem}[Мастер-Теорема]    
Пусть $T(n) = aT(\frac{n}{b}) + n^{c}$, где $a>0, b>1, c\ge0$.\\
Тогда:
\[
    T(n) = \begin{cases}
        \Theta(n^{\log_b a}), & a > b^c.\\
        \Theta(n^c), & a = b^c.\\
        \Theta(n^c \log n),& a < b^c
    \end{cases}
.\] 
\end{theorem}
\begin{example}
    $T(n) = 2T(\frac{n}{2}) + n^2 = n^2 + 2(\frac{n}{2})^2 + 4(\frac{n}{4})^2 + \ldots = n^2(1 + \frac{1}{2} + \frac{1}{4} + \ldots) = n^2 \cdot 2 = \Theta(n^2)$
\end{example}
\begin{proof}
    $T(n) = aT(\frac{n}{b})+n^c=n^c + a (\frac{n}{b})^c + a^2 (\frac{n}{b^2})^c + \ldots = n^ \cdot (1 + \frac{a}{b^c} + (\frac{a}{b^c})^2 + \ldots)$.
    Тогда рассмотрим случаи:
    \begin{itemize}
        \item $a < b^c \Rightarrow \Theta(n^c)$
        \item $a = b^c \Rightarrow \Theta(n^c \log n)$ 
        \item $a > b^c$. Тогда под скобкой получается сумма вида  $1+x+x^2+x^3+\ldots$. Заметим, что сумма равна $\frac{(x^{k+1}-1)}{x - 1}$, где -1 ~-- константа, $x - 1$ ~-- тоже. Тогда сумма равна $\Theta(x^{k+1}) = \Theta(x^k)$. Тогда получаем $\Theta(n^c \cdot (\frac{n}{b^c})^{\log_b n}) = \Theta(n^{\log_b a})$
    \end{itemize}
\end{proof}
\begin{statement}
    $\forall a, b, c > 0\text{:  } \log^a n < n^b < c^n$
\end{statement}
\begin{proof}
    Смотри в конспекте у Сережи.
\end{proof}
