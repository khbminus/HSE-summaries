Рассмотрим простейшую систему уравнений: 
\begin{align*}
    \begin{cases}
        ax + by = e \\
        cx + dy = f.
    \end{cases} \iff x + x \cdot\binom{a}{c} + y \cdot \binom{b}{d} = \binom{e}{f}
\end{align*}
Выразим $\binom{e}{f}$ через $\binom{a}{c},\ \binom{b}{d}$: так как  $x \cdot \binom{a}{c} = \binom{xa}{xc}$, тогда  $\binom{xa}{xc} + \binom{yb}{yd} = \binom{xa + yb}{xc + yd}$.

\begin{definition}
Тогда $x\binom{a}{c} + y\binom{b}{d}$ --- линейная комбинация  $\binom{a}{c}$ и  $\binom{b}{d}$.
\end{definition}
\begin{definition}
    А $\left\{x \binom{a}{c} + y \binom{b}{d}\right\}$ --- линейная оболочка  $\binom{a}{c}$ и  $\binom{b}{d}$. Она обозначается  $\langle \binom{a}{c}, \binom{b}{d} \rangle$.
\end{definition}
\begin{definition}
    Пусть $R$ --- кольцо. \\
    Множество $\left\{ \begin{pmatrix} a_1 \\ a_2 \\ \vdots \\ a_n \end{pmatrix} \mid a_1, a_2, \ldots, a_n \in R\right\}$ --- называется $n$-мерным арифметическим пространством с координатным пространством над $R$.

    С операциями:
    \begin{itemize}
        \item $\begin{pmatrix} a_1 \\ a_2 \\ \vdots \\ a_n \end{pmatrix} + \begin{pmatrix} b_1 \\ b_2 \\ \vdots \\ b_n \end{pmatrix} = \begin{pmatrix} a_1 + b_1\\ a_2 + b_2 \\ \vdots \\ a_n + b_n \end{pmatrix}$ 
        \item $r \cdot \begin{pmatrix} a_1 \\ a_2 \\ \vdots \\ a_n \end{pmatrix} = \begin{pmatrix} ra_1 \\ ra_2 \\ \vdots \\ ra_n \end{pmatrix}$  
    \end{itemize}
\end{definition}
\begin{definition}
    Пусть $K$ --- поле. Векторным пространством над $K$ это тройка  $\left( V, +, \cdot \right)$, где  $V$ --- множество,  $+\!: V \times V \to V$,  $\cdot\!: K \times V \to V$. Причем:
     \begin{enumerate}
         \item [1--4] $(V, +)$ --- абелева группа.
     \setcounter{enumi}{0}
     \item $a + b = b+a\ \forall a, b \in V$.
     \item  $(a+b)+c = a+(b+c)$
     \item  $\exists \overline{0}\!: a + \overline{0} = a$
     \item  $\forall a \in v\ \exists (-a)\!: a+(-a) = \overline{0}$
     \item $(k_1k_2)v = k_1(k_2v)$
     \item $(k_1+k_2)v = k_1v+k_2v$
     \item $k(v_1 + v_2) = kv_1 + kv_2$
     \item $1_K \cdot v = v$
    \end{enumerate}
\end{definition}
\begin{remark}
    $V$ --- векторное пространство над  $K$. Тогда: 
     \begin{itemize}
         \item $0 \cdot v = \overline{0}\ \forall v \in V$.
         \item  $k \cdot \overline{0} = \overline{0}\ \forall k \in K$.
         \item $(-1)\cdot v = -V\ \forall v \in V$.
    \end{itemize}
\end{remark}
\begin{remark}
    Из определений 2--8 следует 1.
\end{remark}
\begin{definition}
    Пусть $R$ --- кольцо. \\
    Тройка  $\left(V, +, \cdot\right)$ с аксиомами 1-8 называется модулем над $R$.
\end{definition}
\begin{remark}
    Абелевы группы $\implies$ модули над  $\Z$.
\end{remark}
\begin{definition}
    $V$ --- векторное пространство над  $K$. $v_1, v_2, \ldots, v_n \in V$. $a_1, a_2, \ldots, a_n \in K$. Тогда $\sum a_i v_i$ --- линейная комбинация  $v_1, v_2, \ldots, v_n$.
\end{definition}
\begin{definition}
    $M \subset V$.  $\langle M \rangle = \left\{ a_1v_1 + a_2v_2 + \ldots + a_k v_k \mid a_i \in K, v_i \in M\right\}$ называется линейной оболочкой множества $M$.
\end{definition}
\begin{definition}
    Подпространство $V$ --- подмножество  $U \subset V$, такое что  $(U, +_V, \cdot_V)$ --- векторное пространство. 
\end{definition}
\begin{statement}
    $U \subset V$ --- подпространство  $\iff$ все операции с элементами  $U$ лежат в $U$.
\end{statement}
\begin{example}
    $\prescript{n}{}{K}$ --- арифметическое пространство.

    $v_1, v_2, \ldots, v_m \in K^n$. $x_1v_1+x_2v_2+\ldots+x_mv_m = 0 = \begin{pmatrix} 0 \\ 0 \\ \vdots \\ 0 \end{pmatrix}$ --- однородная система линейных уравнений. Множество решений $(x_1, x_2, \ldots, x_m) \in \prescript{n}{}{K}$ является подпространством в $\prescript{n}{}{K}$. А дальше я не понял)
\end{example}
\slashn

Обозначение: $U$ --- подмножество  $V$:  $U \le V$.

\begin{statement}
    $V_1, V_2, \le V \implies V_1 \cap V_2 \le V$.
\end{statement}
\begin{proof}
    Очевидно!
\end{proof}
\begin{definition}
   Сумма по Минковскому: $A, B \subset V\!: A + V \coloneqq \{a + b \mid a \in A \land b \in B\}$. 
\end{definition}
\begin{statement}[Сумма по Минковскому]
    $V_1, V_2 \le V \implies V_1 + V_2 \implies V$.
\end{statement}
\begin{proof}
    \slashn
    \begin{itemize}
        \item $x, y \in V_1 + V_2 \iff x = v_1 + v_2, y = v'_1 + v'_2$, где $v_1, v'_1 \in V_1$, $v_2, v'_2 \in V_2$.
        \item $k \cdot x$ --- очевидно.
    \end{itemize}
\end{proof}
\begin{remark}
    $M \subset V$,  $\langle M \rangle = \bigcap\limits_{\mathclap{\substack{U \le V \\ U \supset M}}} U$
\end{remark}
\begin{definition}
    $V_1, V_2$ --- векторные пространствами над $K$. Тогда  $f\!: V_1 \to V_2$ --- гомоморфизм (линейного отображения), если 
    \begin{enumerate}
        \item $f(v_1+v_2) = f(v_1) + f(v_2)\ \forall v_1, v_2 \in V_1$.
        \item $f(kv) = k f(v)$.
    \end{enumerate}

    Если при этом $f$ --- биекция, то  $f$ --- изоморфизм.
\end{definition}
\begin{remark}
    Координизация --- сопоставление элементам векторного пространства координат пространства, являющимся изоморфным этому пространству. 
\end{remark}
\begin{example}[векторных пространств]
\slashn
\begin{enumerate}
    \item $K$ --- векторное пространство над $K$.
     \item Вектора над плоскостью/пространством.
     \item $K\left[x\right]_n = \{ x \in K[x] \mid \deg f \le n \}$. Тогда $K[x]_n \cong K^{n+1}$. 
     \item $M$ --- множество,  $K$ --- поле. Тогда  $V = \{ f\!: M \to K\}$ --- векторное пространство:
         \begin{itemize}
            \item $(f_1 + f_2)(m) \coloneqq f_1(m) + f_2(m)\ \forall m \in M$.
            \item $(kf)(m) = k \cdot f(m)\ \forall k \in K$.
         \end{itemize}
     \item[4'.] $M=K=\R$,  $C_0(\R)$ --- непрерывные функции  $\R \to \R$.  $C_0(\R) \le (a_0, a_1, \ldots)$. Значения во всех рациональных точках.
     \item Последовательность фиббоначиевого типа: $a_n = a_{n-1} + a_{n-2}$. Тогда множество таких последовательностей --- векторное пространство  $\cong \R^2$
     \item $M$ --- множество.  $V = 2^M$,  $K = \Z / 2 \Z$, $M_1 + M_2 \coloneqq M_1 \bigtriangleup M_2,\ 0 \cdot M = \emptyset, 1 \cdot M = M$. Тогда $V$ --- векторное пространство над  $\Z / 2 \Z$,  $V \cong \prescript{n}{}{(\Z / 2 \Z)}$
\end{enumerate}
\end{example}
\begin{definition}
    $V$ --- векторное пространство над  $K$.  $\{ v_i \}_{i \in I}$ называется базисом  $V$, если  $\forall v \in V\ \exists! \{a_i\}_{i \in I} \land a_i \in K\!: v = \sum a_i v_i$
\end{definition}
\begin{remark}
    В терминах этого определения $I = \{1, 2, \ldots, n\}$ $V \leftrightarrow \begin{pmatrix} a_1 \\ a_2 \\ \vdots \\ a_n \end{pmatrix}$, то есть $V \cong K^n$.
\end{remark}
\begin{definition}
    $V$ --- векторное пространство над полем  $K$, тогда  $\{v_i\}_{i \in I}$ называется линейно независимым (ЛНЗ), если выполнено одно из равносильных утверждений:
     \begin{itemize}
         \item $\nexists i \in I\!: V_i = \sum\limits_{j \neq i} a_jv_J$ 
         \item $\forall \{a_i\} \in K\!: \sum a_i v_i = 0 \implies a_i = 0\ \forall i \in I$.
    \end{itemize}
\end{definition}
\begin{proof}
    $2 \implies 1$.  Пусть  $\exists i\!: v_i = \sum a_j v_j \implies \sum a_j v_j - v_i = 0 \xRightarrow{a_i = -1}$ не выполняется второе.

    $1 \implies 2$. Пусть  $a_i v_i = 0$, причем  $\exists a_i \neq 0$. Тогда можно поделить на  $-a_i$ и получить  $v_i = \sum\limits_{i \neq j} b_j v_j$.
\end{proof}
\begin{theorem}[Равносильное определение базиса]
    $\{v_i\}_{i \in I}, v_i \in V$,  $V$ --- векторное пространство над $K$.
     \begin{enumerate} 
         \item $\{v_i\}$ --- базис.
         \item $\{v_i\}$ --- линейно независимая система и  $\langle \{v_i\} \rangle V$.  $\{v_i\}$ --- порождающая система.
         \item  $\{v_i\}$ --- максимально линейная независимая. $\forall v \in V\!: \{v_i\}_{i \in I} \cup v$ --- линейно зависимая.
         \item  $\{v_i\}$ --- минимальная порождающая система. То есть выкидывание любого вектора делало систему не порождающей.
    \end{enumerate}
\end{theorem}
\begin{proof}
    \slashn
     \begin{itemize}
         \item $1 \implies 2$.  $\{v_i\}$ --- базис  $\implies \{v_i\}$ порождающая по определению. Причем  $\sum a_i v_i = 0$ выполняется при  $a_i = 0$, тогда из независимости следует, что  $a_i = 0$.  
         \item $2 \implies 1$.  $\forall v \in V\!: v = \sum a_i v_i$, так как $\{v_i\}$ --- порождающая. Тогда докажем единственность: пусть существуют $\sum a'_i v_i = v = \sum a_i v_i$. Тогда возьмем разность: $0 = \sum (a_i - a'_i) v_i \iff a_i - a'_i = 0 \iff a_i = a'_i$.
         \item  $3 \implies 2$. Раз любой новый вектор добавляет зависимость, то по определению  $V = \sum a_i v_i$. Значит система порождающая. 
         \item $2 \implies 4$. Пусть наша  $\{v_i\}$ --- ЛНЗ. Тогда заметим, что если она не минимальная порождающая, то значит убрав один вектор, мы сможем его получить при помощи других наших векторов  $\Rightarrow$ система линейно зависима.
    \end{itemize}
\end{proof}

\begin{definition}
    $V$ --- векторное пространство над  $K$.

     $V$ называется конечномерным, если  $\exists$ конечная порождающая система  $\in V$.
\end{definition}
\begin{lemma}
    Из любой конечной порождающей системы $V = \langle v_1, v_2, \ldots, v_n \rangle$ можно выбрать базис.
\end{lemma}
\begin{proof}
    Во-первых, если она линейно независима, то все очевидно.

    Тогда, пусть $\exists v_i = \sum\limits_{j \neq i}a_j b_j$. Тогда заметим, что система никак не пострадает, если убрать $V_i$ из системы: мы все равно можем его получить при помощи остальных векторов. 

    Теперь можно продолжить этот процесс до момента, когда эта система станет линейно независимой. Так как система была конечной, то этот процесс когда-либо закончится (например, если выкинем все вектора).
\end{proof}

\begin{remark}
    Пример пространства с нулевым базисом: у множества $V = \{ 0 \}$ базис равен  $\emptyset$.
\end{remark}
\begin{consequence}
    В любом конечном пространстве есть базис.
\end{consequence}
\begin{remark}
    В любом пространстве есть базис.
\end{remark}
\begin{example}
    \begin{align*}
        K[x] = \langle 1, x, x^2,  \ldots \rangle \\
        K[[x]] = \langle ??? \rangle
    \end{align*}

    У $K[[x]]$ есть базис, но на человеческом нельзя задать.
    
    У $\R$ тоже есть базис, но как его задать --- вопрос.
\end{example}
\begin{definition}
    Размерность пространства $\dim V$ --- количество элементов в базисе.
\end{definition}
\begin{theorem}
    Все базисы имеют поровну элементов.
\end{theorem}
\begin{lemma}[Лемма о ЛЗЛК]
    $u_1, \ldots, u_n \in \langle v_1, v_2, \ldots, v_n \rangle$, $m > n$. Тогда  $u_1, \ldots, u_m$ линейно зависима.
\end{lemma}
\begin{proof}
    \begin{lemma}
        Лемма о замене: $\langle v_1, v_2, \ldots v_n \rangle = \langle \sum a_i v_i, v_2, \ldots, v_n \rangle$.
    \end{lemma}
    \begin{proof}
        $\sum a_i v_i, v_2, \ldots, v_n \in \langle v_1, v_2, \ldots, v_n \rangle$. В обратную сторону: ???.
    \end{proof}
    
    Если у нас есть нуль-вектор, то мы сразу проиграли. Иначе представим $\overline{0} = u_1 = \sum a_i v_i$. По лемме произведем замену. И так пока система линейно зависима. А дальше я выпал :(
\end{proof}
