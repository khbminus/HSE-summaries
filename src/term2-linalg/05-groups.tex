Теория групп = теория неабелевых групп, теория абелевых групп $\approx$ линейная алгебра.

Например,  $G=\langle x_1, x_2, \ldots, x_n \rangle$. Тогда $G \cong$ произведению циклических групп. 

Или $G = \langle a, b \rangle$,  $G$ --- абелева, тогда  $G = \{ a^k b^l \mid k, l \in \Z\}$. 

\begin{definition}
    $V$ --- конечное множество,  $|V| = n$.  $S(V) = \{f\!: V \to V (f\text{ --- биекция})\}$.

    Если $V = 1..n \implies S(V) = S_n$.
\end{definition}
\begin{definition}
    $K$ --- поле,  $n \in \N$, тогда  $GL(n, K)$ --- обратимые матрицы порядка  $n$ $= \{A\!: V \to V \mid \small \begin{array}{l} A\text{ --- биективное отображение}\\V\text{ --- }n\text{-мерное пространство}\end{array}\}$
\end{definition}
\begin{remark}
    $S_n$ вкладывается в  $GL(n, K)$.
\end{remark}
\begin{proof}
    $\pi$ --- перестановка,  $V = \langle e_1, e_2, \ldots, e_n \rangle$. $A_\pi(e_i) = e_{\pi(i)}$. Тогда  $\pi \to A_\pi$ --- инъективный гомоморфизм.
\end{proof}
\begin{theorem}[Теорема Кэли]
    Любая конечная группа изоморфна подгруппе в $S_n$ при некотором  $n$.
\end{theorem}
\begin{proof}
    Положим $n = |G|$.  $G$ --- конечная группа, то есть  $G = \{ g_1, \ldots, g_n\}$. Тогда $g \in G\quad m_g\!: G \to G \quad m_g(g_i) = gg_i$ --- биекция.

    То есть  $g \cdot g_i = g_{\pi_g(i)}, \pi_g \in S_n$ --- перестановка.

    Теперь зададим гомоморфизм:  $i\!: \begin{array}{l} G \to S_n \\ g \to \pi_g \end{array}$.  $i$ --- инъективно:  $\pi_g = \pi_{g'} \implies g \cdot e_G = g' \cdot e_G \implies g = g'$.

    Покажем, что $i$ --- гомоморфизм: надо проверить  $\pi_{g_1g_2} = \pi_{g_1} \cdot \pi_{g_2}$.

    $\pi_{hf} = \pi_h \cdot \pi_f$. \quad $g_{\pi_h(\pi_f(i))} = h \cdot g_{\pi_f(i)} = h(fg_i) = (hf)g_i = g_{\pi_{hf}(i)}$.
\end{proof}
\begin{remark}
    Заметим, что в доказательстве теоремы Кэли мы находим не обязательно минимальное регулярное представление. Например, для $S_4$ минимальное представление равно  $S_4$, а у нас  $S_{24}$.
\end{remark}
\Subsection{Смежные классы и теорема Лагранжа}
Пусть есть группа $G$ и  $H \le G$ --- подгруппа.

\begin{definition}
    Левый смежный класс по подгруппе $H$, это  $gH = \{g\cdot h \mid h \in H\}$.
\end{definition}
\begin{definition}
    Правый смежный класс по подгруппе $H$ ---  $H \cdot g = \{ h \cdot g \mid h \in H\}$
\end{definition}
Вообще говоря $gH \neq Hg$ (если $H$ --- неабелева).
\begin{example}
    Пусть $g \in H \implies gH = Hg=H$.
\end{example}
\begin{properties}[смежных классов]
    \begin{enumerate}
        \item $g_1H = g_2H \iff g_2^{-1}g_1 \in H \iff g_1^{-1}g_2 \in H$.
        \item $\forall 2$ смежных класса не пересекаются или совпадают.
    \end{enumerate}
\end{properties}
\begin{proof}
    \begin{enumerate}
        \item $g_1H = g_2H \iff \{g_1h\} = \{g_2h\} \iff \{g_2^{-1}g_1h\} = \{h\} = H$.

            $\Leftarrow\!:$ $g_2^{-1}g_1 \in H \implies \{g_2^{-1} g_1h\} = H$.

            $\Rightarrow\!:$ $\{g_2^{-1}g_1h\} = H$, подставим $h=e \implies g_2^{-1}g_1 \in H$.

            Аналогично для  $g_1g_2^{-1}$. Получили классы эквивалентности: $g_1 \underset{H}{\sim} g_2 \iff g_1H = g_2H$.
        \item Докажем отношение эквивалентности: $g_1 \underset{H}{\sim}g_2 \iff g_1 \in g_2H$.

            $\Rightarrow$.  $g_1H=g_2H$. Подставим $h=e$.

            $\Leftarrow$.  $g_1 = g_2 \cdot h_0, h_0 \in H$. $\{ g_1h \mid h \in H\} = \{(g_2h_0)h\mid h\in H\} = \{g_2(h_0h) \mid h\in H\} = \{g_2 \widetilde{h} \mid \widetilde{h} \in H\} = g_2H$.
    \end{enumerate}
\end{proof}
\begin{remark}
    Схожие утверждения верны и для правых классов.
\end{remark}
Итого: $\forall$ подгруппа  $H$ задает 2 разбиения  $G$ на смежные классы.
\begin{example}
    $G=(\Z, +), H = \langle a \rangle = \{ ka \mid k \in \Z\}$

    $l = \{l + ka \mid k \in \Z\} = \overline{l_a}$ ---  $a$ классов вычетов.
\end{example}
\begin{definition}
    $G$ --- группа,  $H$ --- подгруппа, такая что  $\exists k$ левых смежных классов,  $k$ называется индексом  $H$ в  $G$. Обозначение: $|G : H| = k$
\end{definition}
\begin{exerc}
    $gH \longleftrightarrow Hg^{-1}$ --- биекция между левыми и правыми смежными классами.
\end{exerc}
\begin{theorem}[Теорема Лагранжа]
    $|G| = n, |H| = k$,  $H \le G$.

    Тогда $|G:H| = \frac{n}{k}$. В частности $|G| \divby |H|$. Порядок группы делится на порядок подгруппы.
\end{theorem}
\begin{proof}
    Вспомним доказательство частного случая с первого модуля (порядок группы делится на порядок элемента). Мы фиксировали $a$ и разбивали все элементы на циклы длины $\ord_G a$ вида $x \to ax \to a^2x \to \ldots a^{\ord_G(a) - 1}x \to x$. В новых терминах получившиеся циклы --- классы смежности $a$ по подгруппе $H$. $\forall g \in G\!: |gH| = |H| = \ord a = k \Rightarrow |G : H| = \frac{n}{k}$
\end{proof}
\begin{definition}
    $G / H$ --- множество левых смежных классов.

    $H \setminus G$ --- множество правых смежных классов.
\end{definition}
\Subsection{Группа перестановок}
\begin{definition}
    $\pi \in S_n$ называется циклом, если  $\exists i_1, \ldots, i_k \in \{1..n\}$, такое что $\pi(i_l) = i_{l+1}$и  $\pi(i_k) = i_1$ и  $\pi(j) = j$ для остальных.
\end{definition}
\begin{definition}
    Циклы называются независимыми, если их множества подвижных точек не пересекаются.
\end{definition}
\begin{remark}
    $\pi_1, \pi_2, \ldots, \pi_n$ --- попарно независимые циклы $\implies$ их произведение не зависит от порядка множителей.
\end{remark}
\begin{theorem}
    $\pi \in S_n$.  $\pi$ --- единственным образом (с точностью до порядка) представима как произведение независимых циклов.
\end{theorem}
\begin{proof}
    Будем доказывать для $S_n \cong S(M)$.

     \begin{itemize}
         \item База, $n = 1$.
         \item Переход: $1,\ldots, n-1 \to n$.

             Рассмотрим $\pi \in S(M), a \in M$. Рассмотрим  $\pi(a), \pi(\pi(a)), \ldots$. Рассмотрим минимальное $k$, такое что  $\pi^k(a) = \pi^l(a)$ для какого-то  $0 \le l < k$./

             Если $l \neq 0$, так как  $\pi$ --- биекция, то  $\pi^{k-1}(a) = \pi^{l-1}(a)$. Противоречие.

             Если  $l = 0$, то получили цикл.  $N = \{ \pi^{i}(a) \mid i < k \}$. Пусть  $\pi_0(x) = \pi(x)$, если  $x \in N$ или  $x$ иначе. По индукционному предположению существуют циклы.
    \end{itemize}

    Единственность: ему лень :(
\end{proof}

\begin{definition}
    $\pi$ --- цикл на  $i_1, i_2, \ldots, i_k$. $\pi = (i_1 i_2..i_k)$

    Тогда $\pi \in S_n$ --- произвольная перестановка,  $\pi = (i_1 \ldots i_k)(j_1 \ldots j_l) (s_1 \ldots s_m)$
\end{definition}

\begin{theorem}
    $(ij)$ --- транспозиция. $\langle (ij) \rangle = S_n$.
\end{theorem}
\begin{proof}
    Очев $+$ доказывали
\end{proof}
\begin{theorem}
    $\langle (ijk) \rangle = A_n$ --- группа четных перестановок.
\end{theorem}
\begin{proof}
    $(ijk) = (ij)(jk) \implies (ijk) \in A_n \implies \langle (ijk) \rangle \le A_n$.

    Обратно: пусть $\pi \in A_n$.  $\pi = t_1t_2\ldots t_{2k}$, $t_i$ --- транспозиции. Достаточно доказать, что  $\forall $ транспозиций  $t_i, t_j$  $t_i \cdot t_j \in \langle (ijk)\rangle$.

    Пусть  $t_i = (ab), t_j = (cd)$. Тогда рассмотрим 3 случая:
     \begin{enumerate}
         \item $t_i = t_j \implies t_i \circ t_j = t_i^2 = \mathrm{id} \in \langle (ijk) \rangle$.
         \item  $b = c$,  $a \neq d$. Очев.
         \item $a, b, c, d$ различны. Легко показать, что  $(ab)(cd) = (cad)(abc) \in \langle (ijk) \rangle$.
    \end{enumerate}
\end{proof}

\begin{definition}
    $a \underset{H}{\equiv} b \Leftarrow a \in bH$ (далее $\equiv$ вместо $\sim$).
\end{definition}
\begin{example}
   $H = \{ n\Z \mid x\in \Z\}$ --- подгруппа.

   $b \cdot H = b + H$ --- класс вычетов по модулю  $n$. $a \in bH$  $a = b\cdot h, h \in H \iff b^{-1}a \in H$.

   $a \underset{H}{\equiv}b$, если  $a \in Hb$ ($ab^{-1} \in H$).
\end{example}
\Subsection{Факторгруппа}
\begin{definition}
    $H \le G$ называется нормальной ($H \trianglelefteq G$), если выполнено любое из трех равносильных утверждений:
    \begin{enumerate}
        \item $a \underset{H}{\equiv} b, c \underset{H}{\equiv} d \implies ac \underset{H}{\equiv} bd \quad \forall a,b,c,d \in G$.
        \item $\forall a \in G\!: aH=Ha$.
        \item  $\forall h \in H, g \in G\!: g^{-1}hg \in H$.
    \end{enumerate}
\end{definition}
\begin{definition}
    $g^{-1}hg$ и  $h$ называются сопряженными посредством  $h$.
\end{definition}
\begin{proof}
    Направления доказательства $3 \implies 1$.

     $a \equiv b \implies a = bh_1, c \equiv d \implies c = dh_2$, где $h_1, h_2 \in H$ $\implies ac = bh_1dh_2 = bdd^{-1}h_1dh_2 = bd \cdot h_3 \cdot h_2$, $h_3, h_2 \in H \implies \in bdH$.
\end{proof}
\begin{remark}
    Сопряженность --- отношение эквивалентности.
    $G = \bigsqcup\limits_i C_i$,  $C_i$ --- класс сопряженности.
\end{remark}
\begin{definition}
    $H \trianglelefteq G$. Факторгруппой  $G / H$ называется множество классов смежности со следующей операцией $(a\cdot H)(b\cdot H) = ab \cdot H$.

    Обозначение: $\overline{a_H}$,  $\overline{a_H} \cdot \overline{b_H} \coloneqq \overline{ab}_H$. Заметим, что первое определение нормальности показывает корректность данной операции.
\end{definition}
\begin{example}
    $G = S_n$.  $H \trianglelefteq S_n \implies \left[ \begin{array}{l} H = \{e\} \\ H = S_n \\ H = A_n \end{array} \right.$

     $G / G = \{ \overline{e} \}$,  $G / \{e\} \cong G$.

     $S_n / A_n$:  $e \cdot A_n = A_n$,  $(12) \cdot A_n = S_n \setminus A_n$.
\end{example}

Напоминание: ($K$ --- поле) $GL_n(K)$ --- обратимые матрицы размера  $n$,  $SL_n(K) = \{ A \in M_n(K) \mid \det A = 1\}$.

\begin{statement}
    $SL_n(K) \trianglelefteq Gl_n(K)$.

     $GL_n(K) / SL_n(K)$.  $A \cdot SL_n(K) = \{B \mid \det B = \det A \}$.

     Поэтому  $GL_n(K) / SL_n(K) \cong K^*$.
\end{statement}

\begin{statement}
    $T$ --- множество диагональных матриц, причем  $a^n = 1$. Тогда  $T \trianglelefteq SL_n(K), SL_n(K) / T = PSL_n(K)$ --- projective special group.
\end{statement}
\begin{example}
    $f = \{ \begin{array}{l} ax + b \\ cx + d \end{array} \mid ad - bc \neq 0 \}$ --- группа дробно-линейных преобразований над $K$. $f \leadsto = \begin{pmatrix} a & b \\ c & d \end{pmatrix} = A$,  $f_1 \leadsto \begin{pmatrix} a_1 & b_1 \\ c_1 & d_1 \end{pmatrix} = A_1$, $f \circ f_1 \leadsto A \cdot A_1$. То есть группа дробно-линейных преобразований ---   $PGL_2(n)$.
\end{example}
\Subsection{Теорема о гомоморфизме}
Пусть $G_1, G_2$ --- группы. $f\!: G_1 \to G_2$ --- гомоморфизм, если $f(g_1g_2) = f(g_1)f(g_2)$. Изоморфизм $\iff$ гомоморфизм + биекция.  $f$ --- автоморфизм  $\iff$ изоморфизм и  $G_1 = G_2$.
\begin{remark}
    $G$ --- абелева,  $f(g) = g^{-1}$ --- автоморфизм.

     $g_0 \in G$ --- фиксированно, $f_2(g) = g_0^{-1}gg_0$ --- автоморфизм сопряжения.
\end{remark}

$\ker f = \{g \in G \mid f(g) = e_{G_2}\}$ --- ядро гомоморфизма.  $\Im f = \{ f(g) \mid g \in G \}$ --- образ гомоморфизма.

\begin{lemma}
    $f\!: G_1 \to G_2$, $\Im f \le G_2, \ker f \trianglelefteq G_1$.
\end{lemma}
\begin{proof}
    $f(g_1) = e, f(g_2) = e$. $f(g_1g_2) = f(g_1)f(g_2) = e$.
    $f(e_{G_1}) = e_{G_2}, f(e) = f(e \cdot e) = f(e) \cdot f(e)$.

    $h \in \ker f \Rightarrow f(g^{-1}hg) = f(g^{-1}) f(h)f(g) = f(g^{-1}) f(g) = e$.

    P.S. Тут не хватает ещё проверок, но они тривиальны
\end{proof}
\begin{theorem}
    $f\!: G_1 \to G_2 \implies G_1 / \ker f \cong \Im f$.
\end{theorem}
\begin{proof}
    Возьмем $a \in \Im f$. Рассмотрим  $f^{-1}(a) = \{b \mid f(b) = a\}$. Возьмем $b_0 \in f^{-1}(a)$, тогда  $b \in f^{-1}(a) \iff f(b) = f(b_0) \iff f(bb_0^{-1}) = e \iff bb_0^{-1} \in \ker f \iff b \in b_0 \cdot \ker f$.

    $f^{-1}(a) = b_0 \ker f$.

    Построим  $\widetilde{f}\!: G_1 / \ker f  \to \Im f$. $\widetilde{f}(b \ker f) = f(b)$.
    \begin{enumerate}
         \item $\widetilde{f}$ корректно.  $c \in b \ker f \implies f(c) = f(bh) = f(b) f(h) = f(b)$.
         \item $\widetilde{f}$ --- гомоморфизм:  $\widetilde{f}(\overline{a} \cdot \overline{b}) = \widetilde{f}(\overline{ab}) = f(ab) = f(a)f(b) = \widetilde{f}(\overline{a}) \widetilde{f}(\overline{b})$.
         \item  $\widetilde{f}$ --- сюръективно.  $a \in \Im f \implies a = f(b) \implies a = \widetilde{f}(\overline{b})$.
         \item  $\widetilde{f}$ --- инъективно.  $\widetilde{f}(\overline{b}) = e \iff f(b) = e \iff b \in \ker f \iff \overline{b} = \overline{e}$.
    \end{enumerate}
\end{proof}
\begin{example}
    $G = \R^*, H = \R_+^*$.  $G / H \cong \{1, -1\} \cong \Z / 2\Z$.
     $f\!: G \to G$,  $f(x) = \frac{x}{|x|} = \text{sgn}(x)$.
\end{example}
\begin{example}
    $G = D_4$ --- группа самосовмещений квадрата. $|D_4| = 8$. Есть $1$, 3 поворота и 4 оси симетрии.

     $|G / H| = 4, G / G = F_2^2$. Первый бит --- поворот на $\frac{\pi}{2}$ и зеркаливание.
\end{example}
\begin{example}
    $G_1 / G_2$ --- группы. $G = G_1 \times G_2$, $\widetilde{G_1} = \{ (g_1, 3) \mid g_1 \in G_1 \} \cong G_1$.

    $G / G_1 \cong G_2$. $f\!: G_1 \times G_2 \to G_2$, $(g_1, g_2) \to g_2$, $\ker f = \widetilde{G_1}, \Im f = G_2 \implies G / G_1 = G_2$.
\end{example}

Пусть $G$ --- большая группа. Возьмем  $H \trianglelefteq G$, заменим  $G$ на  $(H, G / H)$. Например, может оказаться, что  $G \cong H \times G / H$.

Пример, $H \cong \Z / 2 \Z, G / H \cong \Z / 2 \Z$, то тогда либо  $G \cong \Z / 2 \Z \times \Z / 2 \Z$ и  $G \cong \Z / 4 \Z$.
\begin{definition}
    $G$ --- называется простой, если у нее нет нетривиальных нормальных подгрупп.
\end{definition}
\begin{theorem}
    $A_n$ --- проста ($n \ge 5$).
\end{theorem}
\begin{theorem}
    $PSL_n(K)$ проста для большинства  $n$.
\end{theorem}
\begin{theorem}
    \slashn
    $G$ --- конечная простая  $\implies \left[\begin{array}{ll} G \cong \Z / p \Z & p\text{ --- простое} \\ G \cong A_n & n \ge 5 \\ G \cong PSL_n(K) & K\text{ --- конечное поле} \\ \text{еще несколько матричных групп}&\\\text{еще 26 исключительных простых групп}&\end{array}\right.$
\end{theorem}
\Subsection{Действие групп}
\begin{definition}
    $G$ --- группа,  $M$ --- множество.

    Действие  $G$ на $M$ --- отображение из  $G \times M \to M$.  $(g, m) \to g \cdot m$, такая что
     \begin{enumerate}
         \item $(g_1g_2) m = g_1(g_2m)$ и $em = m$.
    \end{enumerate}
\end{definition}
\begin{definition}[Альтернативное определение]
    $G$ действует на  $M$, если задан гомоморфизм  $f\!: G \to S(M)$.
\end{definition}
\begin{remark}
    Построим $f_g\!: M \to M, m \mapsto g \cdot m$. Это биекция.
\end{remark}

Говорят $g$ действует на  $M$.  $M$ ---  $G$-множество.  $G \acts M$.
\begin{example}
    \begin{enumerate}
        \item $I_n = \{1..n\}$,  $G = S_n$.  $G \acts I_n$.
        \item  $I_n \times I_n \quad S_n \acts I_n \times I_n \quad \pi(x, y) = (\pi(x), \pi(y))$
        \item  $M=2^{I_n}$.  $S_n \acts 2^{I_n}$.  $\pi \cdot \{x_1, \ldots, x_n\} = \{\pi(x_1),\ldots, \pi(x_n)\}$.
        \item $K$ --- поле.  $K^*$ действует на  $K$ гомоморфизмами.

            ВАЖНО!!!!!!!  $a\cdot b = ab$.
        \item  $(\langle g \rangle)C_n \acts \CC$  $g \cdot z = e^{\frac{2\pi i}{n}} \cdot z$ --- поворот на $\frac{2\pi}{n}$. $g^{k}z = e^{\frac{2\pi i k}{n}z}.$
        \item $S_3 \acts \CC$.  $(123)$ --- умножение на  $-\frac{1}{2} + \frac{\sqrt{3}}{2}i$. $(12) \cdot z = \overline{z}$.
    \end{enumerate}
\end{example}
\begin{definition}
    $G$ точно действует на  $M$, если  $G \to S(M)$ --- инъективно.

    Неточное действие: $G \acts M$  $g \cdot m \coloneqq m$ --- тривиальное действие.
\end{definition}

$S_3 \acts M$ изометриями (точно).  Не $\exists$ точного действия на  $S_4 \acts M$.
\begin{example}[Примеры из теории групп]
Действие сдвигами $G \acts G\!: a \cdot b = a \cdot b$.

 $H \le G$. $G \acts G / H$,  $g_1 (g_2H) = (g_1g_2)H$.
 $\forall $ действие на чтобы то не было сводится к тому, что выше.

  $G \acts G$ сопряжениями:  $a \cdot b = aba^{-1}$. Действие автоморфизмами  $g_1(b) = aba^{-1}$.
\end{example}

\Subsection{Орбиты и стабилизаторы}
\begin{definition}
    $G \acts M$.  $m_1 \sim m_2 \Leftarrow \exists g\!: gm_1 = m_2$.
\end{definition}
\begin{statement}
    $\sim$ --- отношение эквивалентности.
\end{statement}
\begin{definition}
    Класс эквивалентности $\sim$ называется орбитой действия.
\end{definition}
\begin{definition}
    Орбита элемента --- $G \cdot m = \{g \cdot m \mid g \in G\}$.
\end{definition}
 \begin{remark}
    $M$ --- дизъюнктное объединение орбит.
    $M = \bigcup\limits_{i \in I} O_i$ --- орбиты.

    тогда $O_i$ ---  $G$-множества.  $x \in O_i, g \in G \implies gx \in O_i$ --- транзитивные множества.
\end{remark}
\begin{definition}
    Множество называется транзитивным, если $\forall m_1, m_2 \in M \exists g \in G\!: gm_1 = m_2$.
\end{definition}
\begin{definition}
    Iso$(2)$ --- группа изометрий плоскости. Транизитивно: на точки/прямые, не транзитивно на отрезки.
\end{definition}
\begin{definition}
    $G \acts M$,  $m \in M$. Стабилизатор  $G_m = \{g \in G \mid gm = m\}$.
\end{definition}
\begin{example}
    $S_4 \acts 2^{I_4}$.  $m = \{1,2\}, G \cdot m = \{ \{1,2\}, \{1,3\}, \{1,4\}, \{2,3\},\{2,4\},\{3,4\}\}, G_m = \langle (34), (12) \rangle$.
\end{example}
\begin{theorem}
    \begin{enumerate}
        \item $G_m \le G$.
        \item $m \in M, ~ n = g_0m \in Gm, ~ m=g_1n$.  Тогда $\exists$ биекция $Gm \leftrightarrow G/G_m$, причем $\{g \mid gm = n\} = g_0 G_m$, $\{g \mid gn = m \} = G_mg_1$.
        \item $|Gm| \cdot |G_m| = |G|$.
    \end{enumerate}
\end{theorem}
\begin{proof}
    \begin{enumerate}
        \item Очев.
        \item $g_0G_m \subset \{g \mid gm = n\}$ --- ясно. $\supset$: пусть  $gm = n$.  $g_0^{-1}gm = g_0^{-1}n = m \implies g_0^{-1}g \in G_m \implies g \in g_0G_m$.
        \item из $2)$ биекция: элементы орбиты $M \leftrightarrow $ смежные классы  $\implies |Gm| = |G : G_m| \implies |Gm|\cdot |G_m| = |G:G_m||G_m| = |G|$.
    \end{enumerate}
\end{proof}
\begin{example}
    $Iso(2)$ --- движения плоскости.  $S(K) = \{g \in Iso(2) \mid g(k) = k\}$,  $k$ --- квадрат.

    $g \in S(k)$ --- переставляет  $A, B, C, D$. Такая перестановка однозначно задает  $g$.
    $G \cdot A = \{A, B, C, D\}, |GA| = 4$.  $|G| = |G \cdot A| \cdot |G_A| = 4 \cdot |G_A| = 4 \cdot |(G_A)_B| = 8$.

     $G_A \acts \{B,C,D\}$.  $G_A \cdot B = \{B, D\}$.
\end{example}
\Subsection{Лемма Бернсайда}
\begin{definition}
    $Fix(g) = \{m \in M \mid gm = m\}$ --- фиксатор (множество неподвижных точек).
\end{definition}
\begin{theorem}[Лемма Бернсайда]
    $G$ --- конечная,  $G \acts M$. Тогда количество орбит действия равно среднему арифметическому размера фиксатора.
     \[
    \frac{\sum_{g \in G} |Fix(G)|}{|G|} = \text{количество орбит}
    .\]
\end{theorem}
\begin{proof}
    Рассмотрим $NotMove = \{(g, m) \mid g \in G, m \in M, gm = m\}$. Тогда  $\sum\limits_{m \in M}|G_m| = |NotMove| = \sum_{g \in G}|Fix(g)|$. Тогда: \[
    \frac{\sum_{g \in G}|Fix(g)|}{G} = \frac{\sum_{m \in M}|G_m|}{|G|} = \sum_{m \in M} \frac{|G_m|}{|G|} = \sum_{m \in M} \frac{1}{|Gm|}
    .\]
    Тогда, если рассмотреть орбиту длины $k$, то она дает вклад  $\underbrace{\frac{1}{k} + \frac{1}{k} + \ldots + \frac{1}{k}}_{k} = 1 \implies$ сумма количества орбит.
\end{proof}
\begin{example}
    Подсчет числа структур с точностью до изоморфизма.

    Закрепленное ожерелье: $F\!: \Z / 12\Z \to \{B, W\}$. Таких  $2^{12}$.
     $F_1 \sim F_2 \Leftarrow F_1(x) = F_2(x + x_0)$ при некотором $x_0$.

     Пусть  $O$ --- множество закрепленных ожерелий.  $\Z / 12\Z \acts O$. Каково число орбит?  Оно равно  $\frac{|Fix(0)| + |Fix(1)| + \ldots}{12} = $.

     $|Fix(0)| = 2^{12}$,  $|Fix(1)| = |Fix(11)| = |Fix(5)| = Fix(7) = 2^1$,  $|Fix(2)| = |Fix(10)| = 2^2, |Fix(3)| = |Fix(9)| = 2^3, |Fix(4)| = 2^4 = |Fix(8)|, |Fix(6)| = 2^6$.
\end{example}
\Subsection{Применения теории конечных групп действий}
$|G| = n$. Верно ли, что в  $G$ есть элемент порядка  $d$, $n \divby d$? Нет, иначе  бы в группе всегда был элемент порядка $n$, а значит любая группа была бы циклической, что неверно. Но если $d = p$, то ок.

Верно ли, что в  $G$ есть подгруппа порядка  $d$? Нет, например, в  $A_4$ нет подгруппы порядка $6$. Но если $d = p^k$, то ок.

\begin{theorem}[Теорема Коши]
    $|G| \divby p, p$ --- простое $\Rightarrow \exists g \in G\!: \ord g = p$
\end{theorem}
\begin{proof}
    Рассмотрим $M = \{(g_1, \ldots, g_p) \mid g_i \in G, g_1g_2\ldots g_p = e \}$. $|M| = |G|^{p-1} \divby p$, поскольку мы можем выбрать первые $p-1$ элементов произвольным образом а последним взять обратный к произведению. $C_p$ --- циклическая группа порядка $p$, $C_p \acts M\!: t(g_1, g_2, \ldots, g_p) = (g_2, \ldots, g_p, g_1) \in M, \quad t^k(g_1, g_2, \ldots, g_p) = (g_{k+1}, g_{k+2}, \ldots, g_p, g_1, g_2, \ldots, g_k) \in M$.

   Пусть $x \in M$. $|C_p \cdot x| = \frac{|C_p|}{|C_{p_x}|} = \left[ \begin{array}{l} 1 \\ p \end{array} \right.$ ($|C_p| = p$ --- простое). $|M| = \sum \text{длина орбиты} = 1+1+\ldots+1 + p + \ldots + p = a \cdot 1 + b \cdot p \divby p \Rightarrow a \divby p$. Длина орбиты равна $1$ только у элементов из $\{(g, g, \ldots, g) \mid g^p = e\}, \quad \{g \mid g^p = e\} = \{e\} \cup \{g \mid \ord g = p\}$. Поскольку длина орбиты $(e, e, \ldots, e)$ равна $1$, то $a \neq 0 \Rightarrow a \geq p$. Т.е. существует ненулевое делящееся на $p$ количество решений уравнения $x^p = e$, а значит существуют элементы порядка $p$
\end{proof}
\begin{theorem}[Первая теорема Силова]
    Пусть $|G| = p^n \cdot d, d \centernot\divby p$.

    Тогда  $\exists H \le G\!: |H| = p^n$.
\end{theorem}
\begin{proof}
    $M = \{x \subset G \mid |x| = p^n \}$.  $G \acts M$ (сдвигами). $|M| = \binom{p^nd}{p^n} = \frac{(p^nd)!}{(p^n)!(p^n(d-1))!} \centernot \divby p \implies$ длина хотя бы одной орбиты не делится на $p \implies \exists O\!: |O| \centernot \divby p$. $O = Gx$,  $|O| = \frac{|G|}{|G_x|} = \frac{p^nd}{|G_x|} \centernot \divby p \implies |G_x| \divby p^n$, но $|G_x| \le^{(*)}p^n \implies |G_x| = p^n$.

    (*) $x = \{a_1, \ldots, a_{p^n}\}$. $g \in G_x \Rightarrow g a_1 = a_i$. Выбор $i$ однозначно определяет $g = a_i a_1^{-1} \implies$ всего $\le p^n$ вариантов.
\end{proof}
