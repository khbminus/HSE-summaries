Теория групп = теория неабелевых групп, теория абелевых групп $\approx$ линейная алгебра.

Например,  $G=\langle x_1, x_2, \ldots, x_n \rangle$. Тогда $G \cong$ произведению циклических групп. 

Или $G = \langle a, b \rangle$,  $G$ --- абелева, тогда  $G = \{ a^k b^l \mid k, l \in \Z\}$. 

\begin{definition}
    $V$ --- конечное множество,  $|V| = n$.  $S(V) = \{f\!: V \to V (f\text{ --- биекция})\}$.

    Если $V = 1..n \implies S(V) = S_n$.
\end{definition}
\begin{definition}
    $K$ --- поле,  $n \in \N$, тогда  $GL(n, K)$ --- обратимые матрицы порядка  $n$ $= \{A\!: V \to V \mid \small \begin{array}{l} A\text{ --- биективное отображение}\\V\text{ --- }n\text{-мерное пространство}\end{array}\}$
\end{definition}
\begin{remark}
    $S_n$ вкладывается в  $GL(n, k)$.
\end{remark}
\begin{proof}
    $\pi$ --- перестановка,  $V = \langle e_1, e_2, \ldots, e_n \rangle$. $A_\pi(e_i) = e_\pi(i)$. Тогда  $\pi A_\pi$ --- инъективный гомоморфизм.
\end{proof}
\begin{theorem}[Теорема Кэли]
    Любая конечная группа изоморфна подгруппе в $S_n$ при некотором  $n$.
\end{theorem}
\begin{proof}
    Положим $n = |G|$.  $G$ --- конечная группа, то есть  $G = \{ g_1, \ldots, g_n\}$. Тогда $g \in G\quad m_g\!: G \to G m_g(g_i) = gg_i$ --- биекция. 

    То есть  $g \cdot g_i = g_{\pi_g}(i), \pi_g \in S_n$ --- перестановка.

    Теперь зададим гомоморфизм:  $i\!: \begin{array}{l} G \to S_n \\ g \to \pi_g \end{array}$.  $i$ --- инъективно:  $\pi_g = \pi_{g'} \implies g \cdot e_G = g' \cdot e_G \implies g = g'$. 

    Покажем, что $i$ --- гомоморфизм: надо проверить  $\pi_{g_1g_2} = \pi_{g_1} \cdot \pi_{g_2}$.

    $\pi_{hf} = \pi_h \cdot \pi_f$.  $h \cdot g_{\pi(f)(i)} = h(fg_i) = (hf)g_i = g_{\pi_{h,f}(i)}$.
\end{proof}
\begin{remark}
    Заметим, что в доказательстве теоремы Кэли мы находило не обязательно минимальное регулярное представление. Например, для $S_4$ минимальное представление равно  $S_4$, а у нас  $S_{24}$.
\end{remark}
\Subsection{Смежные классы и теорема Лагранжа}
Пусть есть группа $G$ и  $H \le G$ --- подгруппа.

\begin{definition}
    Левый смежный класс по подгруппе $H$, это  $gH = \{g\cdot h \mid h \in H\}$.
\end{definition}
\begin{definition}
    Правый смежный класс по подгруппе $H$ ---  $H \cdot g = \{ h \cdot g \mid h \in H\}$
\end{definition}
Вообще говоря $gH \neq Hg$ (если $H$ --- неабелева).
\begin{example}
    Пусть $g \in H \implies gH = Hg=H$.
\end{example}
\begin{properties}[смежных классов]
    \begin{enumerate}
        \item $g_1H = g_2H \iff g_2^{-1}g_1 \in H \iff g_1^{-1}g_2 \in H$.
        \item $\forall 2$ смежных класса не пересекаются или совпадают.
    \end{enumerate}
\end{properties}
\begin{proof}
    \begin{enumerate}
        \item $g_1H = g_2H \iff \{g_1h\} = \{g_2h\} \iff \{g_2^{-1}g_1h\} = \{h\} = H$.

            Тогда если $g_2^{-1}g_1 \in H \implies \{g_2 g_1h\} = H$. Тогда рассмотрим $h=e$.

            Аналогично для  $g_1g_2^{-1}$. Получили классы эквивалентности: $g_1 \underset{\sim}{H} g_2 \iff g_1\in g_2H$.
        \item Докажем отношение эквивалентности: $g_1 \underset{\sim}{J}h_2 \iff g_1 \in g_2H$.

            $\Rightarrow$.  $g_1H=g_2H$. Подставим $h=e$.

             $\Leftarrow$.  $g_1 = g_2 \cdot h_0, h_0 \in H$. $\{ g_1h \mid h \in H\} = \{(g_2h_0)h\mid h\in H\} = \{g_2(h_0h) \mid h\in H\} = \{g_2 \widetilde{h} \mid \widetilde{h} \in H\} = g_2H$.
    \end{enumerate}
\end{proof}
\begin{remark}
    Схожие утверждения верны и для правых классов.
\end{remark}
Итого: $\forall$ подгруппа  $H$ задает 2 разбиения  $G$ на смежные классы.
\begin{example}
    $G=(\Z, +), H = \langle a \rangle = \{ ka \mid k \in \Z\}$

    $e = \{e + ka \mid k \in \Z\} = \overline{e_a}$ ---  $a$ классов вычетов.
\end{example}
\begin{definition}
    $G$ --- группа,  $H$ --- подгруппа, такая что  $\exists k$ левых смежных классов,  $k$ называется индексом  $H$ в  $G$.
\end{definition}
\begin{exerc}
    $Gh \iff Hg^{-1}$ --- биекция между левыми и правыми смежными классами.
\end{exerc}
\begin{theorem}[Теорема Лагранжа]
    $|G| = n, |H| = k$,  $H \le G$.

    Тогда $|G:H| = \frac{n}{k}$. В частности $|G| \divby |H|$. Порядок группы делится на порядок подгруппы.
\end{theorem}
\begin{definition}
    $G / H$ --- множество левых смежных классов.

    $H \setminus G$ --- множество правых смежных классов.
\end{definition}
\Subsection{Группа перестановок}
\begin{definition}
    $\pi \in S_n$ называется циклом, если  $\exists i_1, \ldots, i_k \in \{1..n\}$, такое что $\pi(i_l) = i_{l+1}$и  $\pi(i_k) = i_1$ и  $\pi(j) = j$ для остальных.
\end{definition}
\begin{definition}
    Циклы называются независимыми, если их множества подвижных точек не пересекаются.
\end{definition}
\begin{remark}
    $\pi_1, \pi_2, \ldots, \pi_n$ --- попарно независимые циклы $\implies$ их произведение не зависит от порядка множителей.
\end{remark}
\begin{theorem}
    $\pi \in S_n$.  $\pi$ --- единственным образом (с точностью до порядка) представима как произведение независимых циклов.
\end{theorem}
\begin{proof}
    Будем доказывать для $S_n \cong S(M)$.

     \begin{itemize}
         \item База, $n = 1$. 
         \item Переход: $1,\ldots, n-1 \to n$.

             Рассмотрим $\pi \in S(M), a \in M$. Рассмотрим  $\pi(a), \pi(\pi(a)), \ldots$. Рассмотрим минимальное $k$, такое что  $\pi^k(a) = \pi^l(a)$ для какого-то  $0 \le l < k$./

             Если $l \neq 0$, так как  $\pi$ --- биекция, то  $\pi^{k-1}(a) = \pi^{l-1}(a)$. Противоречие.

             Если  $l = 0$, то получили цикл.  $N = \{ \pi^{i}(a) \mid i < k \}$. Пусть  $\pi_0(x) = \pi(x)$, если  $x \in N$ или  $x$ иначе. По индукционному предположению существуют циклы.
    \end{itemize}

    Единственность: ему лень :(
\end{proof}

\begin{definition}
    $\pi$ --- цикл на  $i_1, i_2, \ldots, i_k$. $\pi = (i_1 i_2..i_k)$
    
    Тогда $\pi \in S_n$ --- произвольная перестановка,  $\pi = (i_1 \ldots i_k)(j_1 \ldots j_l) (s_1 \ldots s_m)$
\end{definition}

\begin{theorem}
    $\langle (ij) \rangle$ --- транспозиция. $\langle (ij) \rangle = S_n$. 
\end{theorem}
\begin{proof}
    Очев.
\end{proof}
\begin{theorem}
    $\langle (ijk) \rangle = A_n$ --- группа четных перестановок. 
\end{theorem}
\begin{proof}
    $(ijk) = (ij)(jk) \implies (ijk) \in A_n \implies \langle (ijk) \rangle \le A_n$.

    Обратно: пусть $\pi \in A_n$.  $\pi = t_1t_2\ldots t_{2k}$, $t_i$ --- транспозиции. Достаточно доказать, что  $\forall $ транспозиция  $t_i, t_j$  $t_i \cdot t_j \in \langle (ijk)\rangle$.

    Пусть  $t_i = (ab), t_j = (cd)$. Тогда рассмотрим 3 случая:
     \begin{enumerate}
         \item $t_i = t_j \implies t_i \circ t_j = t_i^2 = \mathrm{id} \in \langle (ijk) \rangle$.
         \item  $b = c$,  $a \neq d$. Очев.
         \item Легко показать, что  $(ab)(cd) = (cad)(abc) \in \langle (ijk) \rangle$. 
    \end{enumerate}
\end{proof}
