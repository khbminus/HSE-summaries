\Subsection{Определения}
\begin{definition}
    Граф $G$ --- тройка  $(V, E, I)$:
     \begin{enumerate}
        \item $V$ --- конечное множество вершин.
        \item  $E$ --- конечное множество ребер.
        \item  $I\!: E \to \binom{V}{2}$.
    \end{enumerate}
\end{definition}
\begin{definition}
    Концевые вершины ребра --- вершины, которые соединены этим ребром.
\end{definition}
\begin{definition}
    Если два ребра имеют одинаковые концевые вершины, то такие ребра --- кратные (мультиребра).
\end{definition}
\begin{definition}
    Если ребро соединяет вершину с собой, то это ребро --- петля.
\end{definition}
\begin{definition}
    Граф простой --- без петель и мультиребер.
\end{definition}
\begin{definition}
    Степень ребра (валентность) --- количество ребер, исходящих из вершины.
\end{definition}
\begin{theorem}
    Сумма степеней вершин в графе равна удвоенному количеству ребер. То есть $\displaystyle \sum_{v \in V(G)} \deg v = 2 \cdot |E|$
\end{theorem}
\begin{proof}
    Каждое ребро состоит из двух полуребер. Из каждой вершины <<торчит>> $\deg v$ полуребер (принцип биекции). Тогда получили, что  $\sum \deg v = 2|E|$
\end{proof}
\begin{consequence}
    Количество вершин нечетной степени в графе четно.
\end{consequence}
\begin{definition}
    Ребро $e$ инцидентно вершине $u$, если  $u$ --- концевая вершина ребра. 
\end{definition}

\begin{definition}
	Матрица инцидентности --- таблица, где строчки соответствуют вершинам, а столбцы --- рёбрам, а на пересечении столбца и строки стоит 0, если эта вершина не инцидентна этому ребру, иначе то, сколько раз она ему инцидентна (1, если не петля, иначе --- 2).
\end{definition}

Можно заметить, что сумма всех чисел в каждом столбце --- два, а в каждой строке --- степень вершины. Из этого несложно в очередной раз заметить, что суммма удвоенного количества рёбер есть сумма степеней вершин.

\begin{definition}
    Полный граф --- полный простой граф на $n$ вершинах.  $K_n$. Граф, в котором каждая вершина соединена ребром с каждой.
\end{definition}
\begin{definition}
    Дополнением графа $G$ называется граф  $\overline{G}$:  $V(\overline{G}) = V(G)$, а $E(\overline{G}) = \binom{V(G)}{2} \setminus E(G)$ 
\end{definition}

\begin{definition}
    Граф называется двудольным $V(G) = V_1 \cup V_2, V_1 \cap V_2 = \emptyset$. И все ребра ведут из $V_1$ в  $V_2$.
\end{definition}
\begin{definition}
    Полный двудольный граф --- двудольный граф со всеми возможными ребрами. $K_{n, m}$, если в одной доле $n$ вершин, а в другой  $m$.
\end{definition}

\begin{definition}
	Граф <<$k$-мерный куб>> --- $Q_k$, такой граф, что $V$ --- множество бинарных строк длины  $k$.  $E: e = uv \iff$ $u$ и  $v$ отличаются в одном бите. 
\end{definition}
\begin{remark}
    Заметим, что данный граф двудольный: одна доля c четной суммой битов, другая --- с нечетной.
\end{remark}
\begin{definition}
    $P_n$ --- граф <<путь>>. Просто простой путь. Ничего лишнего.
\end{definition}
\begin{definition}
    $C_n$ --- граф <<цикл>>. Простой путь, замкнутый в кольцо.
\end{definition}
\begin{definition}
    Регулярный граф --- граф, в котором степень всех вершин равны. $R$-регулярный граф --- граф, в котором степени всех вершин равны $R$.
\end{definition}
\begin{definition}
    Оргаф (ориентированный граф) $D = (V, E, I)$, где $I: E \to V \times V$.
\end{definition}
\begin{theorem}
    $\displaystyle \sum_{v \in V(G)} \text{indeg}(v) = \sum_{v \in V(G)} = \text{outdeg}(v) = |E(D)|$
\end{theorem}
\begin{proof}
    Очев. Реально очев. Входящих концов у рёбер суммарно столько же, сколько и исходящих.
\end{proof}
\begin{definition}
    Не помню, было ли тут что-нибудь. Если вам кажется, что мы пропустили какое-то определение --- напишите пж.
\end{definition}
\begin{definition}
    Ориентацией графа $G$ называется граф $G$ полученный ориентацией всех ребер графа $G$.
\end{definition}
\begin{definition}
    Граф называется турниром, если он является ориентацией полного графа.
\end{definition}
\begin{definition}
    Две вершины называются смежными, если есть ребро между ними.
\end{definition}

\begin{definition}
    Матрица смежности --- матрица размера $V$ на $V$. $A_{i, j}$ показывает сколько ребер идет из $i$ в $j$.
\end{definition}
\begin{definition}
    Список смежности --- список списков, где для каждой вершины храним выходящие из нее ребра.
\end{definition}
\Subsection{Маршруты, пути, циклы. Связные графы}
\begin{definition}
    Маршрут (walk) --- набор вершин и ребер вида: $v_0, e_0, v_1, e_1, \ldots$, где $v_i$ --- вершины графа, $e_i$ --- ребра графа, причем $e_i = v_{i-1}v_i$
\end{definition}
\begin{definition}
    Путь (trail) --- маршрут без повторяющихся ребер.
\end{definition}
\begin{definition}
    Простой путь (path) --- Путь (сложный) без повтора вершин.
\end{definition}
\begin{definition}
    Вершины $x$ и  $y \in V$ называются связанными, если существует путь, соединяющий $x, y$. 
\end{definition}
\begin{remark}
    Заметим, что свзяность --- отношение эквивалентности. $x \to x$ --- очев, $x\to y = y \to x$. $x\to y, y\to z$, тогда $x\to z = x\to y \cup y\to z$.

    Тогда можно разбить на блоки --- компоненты связности.
\end{remark}
\begin{definition}
    Расстояние $d(x, y)$ --- длина кратчайшего пути из  $x$ в  $y$.
\end{definition}
\begin{definition}
    Диаметр графа --- это расстояние между двумя наиболее удаленными точками.
\end{definition}
\begin{definition}
	$x \in V(G)$ эксцентриситет  $\varepsilon(x) = \max_{y \in V(G)} d(x, y)$
\end{definition}
\begin{definition}
    Радиус $G\ : r(G) = \min_{x \in V(G)} \varepsilon(x)$
\end{definition}
\begin{definition}
    Замкнутый путь --- путь, которого стартовая вершина равна конечной.
\end{definition}
\begin{definition}
    Обхват --- минимальная длина цикла в графе. Если в графе циклов нет, то равен бесконечности.
\end{definition}
\begin{definition}[Теорема Кенига(Kőnig)]
    Граф двудольный $\iff$ в нем нет циклов нечетной длины.
\end{definition}
\begin{proof}
    \slashn
    \begin{itemize}
        \item $\Rightarrow$. Предположим, что есть цикл нечетной длины. Каждое ребро --- переход в другую долю. То есть в стартовую долю мы переходим через четное число ходов. Противоречие.
        \item $\Leftarrow$, нет простых нечетных циклов  $\Rightarrow$ двудольный.

            Будем, считать, что $G$ --- связный. Возьмем вершину  $x$. Поместим все вершины на нечетном расстоянии в правую долю, все остальные в левую. 

            Неправильное решение: Рассмотрим случаи двух смежных вершин на нечетном + нечетном и четном + четном расстоянии. Проблема: путь может быть сложным, поэтому объединение может давать не простой цикл. 

            Правильное решение: пойти читать конспект Омеля, нам очень лень рисовать или писать словами. Поэтому очень жаль:( (НО! если вы сильный, то можете написать)
    \end{itemize}
\end{proof}
\slashn
Проверка на двудольность: для каждой вершины кидаем смежную вершину в другую долю.
\begin{definition}
    Пусть $x, y$ --- вершины орграфа  $D$, то они называются связанными, если существует путь из  $x$ в  $y$ и существует путь из  $y$ в  $x$.
\end{definition}
\begin{remark}
    Такая связанность --- отношение эквивалентности. 
    Блоки --- компоненты сильной связности.
\end{remark}
\begin{definition}
    Компоненты слабой связности --- компоненты связности, если забыть про ориентацию.
\end{definition}
\begin{lemma}
    Пусть $H_1$,  $H_2$ --- компоненты сильной связности. Если существует ребро  $H_1 \to H_2$, то ребра $H_2 \to H_1$ не существует.
\end{lemma}
\begin{proof}
    Предположим, что ребро существует. Тогда получили одну компоненту сильной связности. Противоречие.
\end{proof}
\begin{lemma}
    Я пропустил лемму. Видимо про то, что из компонент сильной связности можно сделать граф без циклов.
\end{lemma}
\begin{proof}
    По предыдущей лемме --- очев.
\end{proof}
\begin{definition}
    Топологическая сортировка --- запись вершин в таком порядке, что все ребра идут слева направо.
\end{definition}
\begin{statement}
    Любой ациклический граф может быть топологически отсортирован.
\end{statement}
\begin{proof}
    Омель так сказал.
\end{proof}
\begin{remark}
    В графе может быть несколько топологических сортировок.
\end{remark}
\begin{consequence}
    В ациклическом графе есть вершина, из которой не выходит ребро.
\end{consequence}
\Subsection{Подграфы. Основные операции над графами}
\begin{definition}
    Граф $H$ называется подграфом графа  $G$, если 
     \begin{enumerate}
         \item $V(H) \subseteq V(G)$
         \item  $E(H) \subseteq E(G)$ 
         \item $I_H\!: E(H) \to V(H) \times V(H)$

             $I_G: E(G) \to V(G) \times V(G)$

              $I_H$ --- сужение  $I_G$.
    \end{enumerate}
\end{definition}
\begin{definition}
    Операция удаления ребра --- $G \setminus e_5$ или $G - e_5$. Просто удаляем ребро. Получаем  $(G', V, E \setminus e_5)$.
\end{definition}
\begin{definition}
    Операция удаления вершины. Удаляем вершину + все смежные вершины. 
    $(Gm V \setminus \{v\}, E \setminus \bigcap_{u \leadsto v} vu)$
\end{definition}
\begin{statement}
    Любой подграф получается удалением ребер и вершин.
\end{statement}
\begin{proof}
    Удалим ребра $E \setminus E'$, удалим вершины  $V \setminus V'$. 
\end{proof}
\begin{definition}
    Если подграф $H$ графа  $G$ можно получить удалением из  $G$ только ребер, то  $H$ --- остовный подграф.
\end{definition}
\begin{definition}
    Если подграф $H$ графа  $G$ можно получить только удалением вершин, то  $H$ --- порожденный или индуцированный подграф.  $V(H)$ --- порождающее множество.
\end{definition}
\begin{definition}
    Операция стягивания ребра. Есть ребро $(u, v)$ удалим, $u, v$ и добавим вершину  $w$, причем  $w$ смежна с  $x$, если  $x$ смежна с  $v$ или  $x$ смежна с  $u$.
\end{definition}
\begin{definition}
    Мост --- ребро, при удалении которого число компонент связности увеличивается.
\end{definition}
\begin{definition}
    Вершина $v$ называется точкой сочленения, если при удалении данной вершины компонента связности распадается на 2 и более.
\end{definition}
\begin{statement}
    Вершина $v$ в связном графе  $G$ является точкой сочленения тогда, и только тогда, когда существуют вершины  $x$ и  $y$, что любой путь из  $x$ в  $y$ проходит через $v$.
\end{statement}
\begin{proof}
    Пусть $v$ --- точка сочленения. Пусть существует путь из  $x$ в  $y$, не проходящий через  $v$. Тогда после удаления  $v$, из  $x$ в  $y$ все еще будет путь  $\Rightarrow$ они в одной компоненты связности.

    В обратную сторону. Удалим $v$, тогда из  $x$ больше нет пути в  $y$. Значит компонента разбилась хотя бы на две компоненты.
\end{proof}

