\def\multiset#1#2{\ensuremath{\left(\kern-.3em\left(\genfrac{}{}{0pt}{}{#1}{#2}\right)\kern-.3em\right)}}
\Subsection{Сшки}
Есть два способа записи цэшек: $C_{n}^k = \binom{n}{k}= \frac{n!}{k! \cdot (n - k)!}$.
Обычно формулы в комбинаторике используются не для подсчетов, а для определения асимптотики/верней оценки и так далее. Например если взять $n = 100$, то уже проблема: $100!$ --- довольно большое число. Но там еще и деление!!! Короче, может получиться небольшое число при больших числах в подсчетах. 

Давайте забудем эту дурацкую формулу и будем использовать рекурренты: легко считать, пишется в миг. $\binom{n}{k} = \binom{n - 1}{k} + \binom{n}{k - 1}, \binom{0}{0} = 1$.
\begin{proof}
    Пусть есть множество из $n$ элементов.     Разобьем все $k$-элементные подмножества на блоки: в одном все без последнего элемента, в другом все с последним.Тогда в первом блоке тогда есть  $\binom{n - 1}{k}$ элементов. В другом $\binom{n - 1}{k - 1}$ элементов. А значит $\binom{n}{k} = \binom{n - 1}{k - 1} + \binom{n - 1}{k}$
\end{proof}
\slashn
Есть пара граничных случаев: $\binom{n}{0} = 1, \binom{n}{k} (n < k) = 0$.
После этого можно сделать треугольник Паскаля:

\begin{center}
\def\N{6}
\tikz[x=0.75cm,y=0.5cm, 
  pascal node/.style={font=\footnotesize}, 
  row node/.style={font=\footnotesize, anchor=west, shift=(180:1)}]
  \path  
    \foreach \n in {0,...,\N} { 
      %(-\N/2-1, -\n) node  [row node/.try]{Row \n:}
        \foreach \k in {0,...,\n}{
          (-\n/2+\k,-\n) node [pascal node/.try] {%
            \pgfkeys{/pgf/fpu}%
            \pgfmathparse{round(\n!/(\k!*(\n-\k)!))}%
            \pgfmathfloattoint{\pgfmathresult}%
             %$\binom{\n}{\k}=$%  
            \pgfmathresult%
        }}};
\end{center}

Рассмотрим решетчатую плоскость (если вы это читаете это и здесь нет картиночки напишите \texttt{@doktorkrab}, чтобы я добавил картиночку). Какое здесь количество путей? Ну $A{n}^k = A_{n-1}^k + A_{n-1}^{k-1}$. А это Сшки.

Теперь посмотрим на сумму на диагонали. Получаем гипотезу: $\sum{m=0}^n \binom{m}{k}=\binom{k}{k} + \binom{k+1}{k} + \ldots + \binom{n-1}{k} + \binom{n}{k} = \binom{n+1}{k+1}$.
\begin{proof}
    По основному комбинаторному тождеству: $\binom{m + 1}{k + 1} = \binom{m}{k + 1}  \binom{m}{k} \Rightarrow \binom{m}{k} = \binom{m + 1}{k + 1} - \binom{m}{k+1}$. Тогда: \[
        \sum_{m=k}^{n} \binom{m}{k} = \underbrace{\sum_{m=k}^n \binom{m+1}{k+1}}_{\binom{n+1}{k+1} + \sum_{m=k}^{n-1} \binom{m+1}{k+1} } - \underbrace{\sum_{m=k}^{n} \binom{m}{k+1}}_{\sum_{m=k+1}^{n} \binom{m}{k+1}}
    .\] 
    Дальше, если, расписать сумму все получится.

    Пусть хочу набрать $k+1$-элементное подмножество из  $n+1$-элементного множества. Пусть мы выбрали последний элемент, тогда у нас есть $\binom{n}{k}$ способов, а если не выбрали, то  $\binom{n}{k+1}$ способов. А по индукции  $\binom{n}{k+1} = \binom{n-1}{k+1} + \binom{n - 1}{k}$. И так далее.
\end{proof}
\slashn
Рассмотрим $\binom{n+m}{k} = \sum_{i=0}^k \binom{n}{i} \cdot \binom{m}{k-i}$
 \begin{proof}
     Рассмотрим два множества: одно $n$-элементное ("мальчики"), другое  $m$-элементное ("девушки"). Тогда пусть мы выбрали  $i$ мальчиков, тогда нам нужно выбрать $k-i$ девушек. 
\end{proof}
\slashn
Мы здесь применили принцип \texttt{double counting}: если мы посчитали что-то двумя способами, то результаты равны.

\Subsection{Биномиальные коэффициенты}
Подробности на втором курсе.

Рассмотрим бином Ньютона: $(x+y)^n = \sum_{k=0}^n \binom{n}{k} x^k \cdot y^{n-k}$
\begin{proof}
    Раскроем скобки в левой части: $(x+y)(x+y)(x+y)\ldots$. Когда у нас $x^k$? Когда мы ровно в  $k$ скобках выбрали  $x$. Сколько способов? Очевидно  $\binom{n}{k}$.
\end{proof}
Частные случаи:
\begin{itemize}
    \item $x=y=1$. Тогда  $2^n = \sum_{k=0}^n \binom{n}{k}$
    
        Рассмотрим множество $\{x_1,x_2,\ldots,x_n\}$. Каждому числу можно сопоставить 0/1 --- берем/не берем. Тогда количество подмножеств --- количество бинарных строчек длины $n$. Такой метод называется биективным: когда мы доказываем, что один объект является биекцией другого, то их количества равны.
    \item $x = 1, y = -1$. Тогда  $0 = \sum_{k=0}^n (-1)^k \binom{n}{k}$ --- количества способов выбрать подмножество четных длин и нечетных длин равны.
\Subsection{Мультимножество}
Хотим посчитать $\multiset{n}{k}$ --- количество  $k$-элементных подмультимножеств.

Пусть $X = [n]$. По принципу биекции найдем сначала $\multiset{n}{k}$ для  $X$, а потом найти для произвольного множества. 

Пусть есть множество  $A$, заменим его на множество  $\{ i + A_i\}$. $\multiset{n}{k} = \binom{n+k-1}{k}$
\end{itemize}

