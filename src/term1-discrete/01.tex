\Subsection{Базовые понятия}
Есть официальный конспект, который будет \href{}{Здесь}.
\begin{definition}
    Множество --- набор различимых между собой по какому-то признаку предметов.
\end{definition}
\begin{definition}
    Предметы входящие в это множество называются его элементами.
\end{definition}
Если мы хотим описать множество, то нужно просто описать предметы этого множества. Например, чтобы задать множество студентов необходимо задать просто студентов.

Есть конечные, счетные, несчетные и целый зоопарк множеств разных мощностей. Самое простое множество --- $\varnothing$, множество ничего не содержащее --- пустое.
\begin{definition}
     $X$ подмножество ($\subseteq$)  $Y$  $\Leftarrow \forall y \in Y: \; y \in X$. \\
     $\varnothing$ и $X$ --- тривиальные, остальные --- нетривиальные.
     все подмножества, кроме $X$ --- собственные.
\end{definition}

\Subsection{Операции с множествами}
\begin{center}
      \renewcommand{\arraystretch}{1.7}
  \large

\begin{tabular}{| c | c | c |}
    \hline 
    \textbf{Символ} & \textbf{Определение} & \textbf{Словами} \\
    \hline
    $\Large{\cap}$ & $A \cap B = \ldots$ & Пересечение множества \\
    \hline
    $\Large{\cup}$ & $A \cup B= \ldots$ & Объединение множеств \\
    \hline
    $\Large{\setminus}$ & $A\setminus B$ = \ldots & Разность множеств \\
    \hline
    $\Large{\bigtriangleup}$ & $A \bigtriangleup B = \ldots$ & Симметрическая разность множеств \\
    \hline
\end{tabular}
\end{center}
\begin{definition}
    Алгебраическая структура --- множество, на котором ввели какую-то операцию.
\end{definition}
\begin{example}
    Пусть заданы несколько множеств:
     \begin{enumerate}
         \item $\exists e: \; a\cdot e = a\; \forall a \in G$
         \item  $\forall a \in G \; \exists a^{-1} \in G: \; a \cdot a^{-1} = a^{-1} \cdot a = e$ 
         \item $\forall a,b,c:\; (a \cdot b) \cdot c = a \cdot (b \cdot c)$
         \item  $\forall a, b \in G a \cdot b = b \cdot a$
    \end{enumerate}
    То это абелева группа и это к алгебре.
\end{example}

А дискретная математика не имеет аксиом, то есть мало чего можно использовать из алгебры / матана. 

Если задать какое-то надмножество $X$ над $A$, то появится операция дополнения:  $A' = X \setminus A$.
Законы Де Моргана:
\begin{theorem}
    $(A \cup B)' = A' \cap B'$
\end{theorem}
\begin{theorem}
    $(A \cap B)' = A' \cup B'$
\end{theorem}

Доказательство смотри в конспекте Омеля, тут мне лень это делать.

\begin{definition}
    Система иножеств --- множество, элементами которого являются множества.
\end{definition}
\begin{definition}
    Семейство множеств --- упорядоченный набор неких множеств $(X_1, X_2,\ldots, X_k)$. Причем множества в наборе могут повторяться.
\end{definition}
\begin{definition}
    Некоторое покрытие множества $X$ системой множеств --- система множеств, объединение элементов которого равняется $X$.
\end{definition}
\begin{definition}
    Разбиение множества $X$ на блоки --- система $(X_1,X_2,\ldots,X_k)$, удовлетворяющая неким условиям:
     \begin{enumerate}
         \item $X = \bigcup_{i = 1}^k X_i$
         \item $\forall i: \; X_i \neq \varnothing$
         \item  $\forall i, j = 1..k: \; X_i \cap X_j = \varnothing$
    \end{enumerate}
\end{definition}
\begin{definition}
    Пара элементов $(x,y)$ --- упорядоченный набор из двух элементов. То есть для $x \neq y$:  $(x, y) \neq (y,x)$
\end{definition}
\begin{definition}[Декартово произведение]
    $X \times Y$ =  $\{ (x, y) \; \vert \; x \in X, y \in Y \}$
\end{definition}
можно ввести понятие <<$n$ки>> --- упорядоченный набор из $n$ элементов. Поэтому можно ввести $A \times B \times C \times \ldots$ и $A^2$,  $A^n$
\begin{definition}
    Отношение между множествами --- некое подмножество декартого произведения этих множеств
\end{definition}
Пусть $\omega$ --- отношение между $X$ и $Y$. Тогда их записывают $X \omega Y$, а отсутствие --- $X \xcancel{\omega} Y$. 
\begin{definition}
    Отношение эквивалентности $(X, \sim)$:
     \begin{enumerate}
         \item $x \sim x \; \forall x \in X$
         \item $x \sim, y \Rightarrow y \sim x \; \forall x, y \in X$
         \item  $x \sim y, y \sim z, \Rightarrow x \sim z\; \forall x, y, z \in X$
    \end{enumerate}
\end{definition}
Пусть $\widetilde{x}= \{ y \in X \; \vert \; y \sim x\}$. 
\begin{property}
пусть $y \in \widetilde{x} \Rightarrow \widetilde{y} = \widetilde{x}$ 
\end{property}

\begin{theorem}
    Разбиение на блоки задает классы эквивалентности.
\end{theorem}
    %Кек.
    \begin{itemize}
        \item $X = \bigcup_{x \in X} \widetilde{x}$
        \item $\widetilde{x} \neq \varnothing$, т.к. хотя бы $x \in \widetilde{x}$.
        \item Рассмотрим  $\widetilde{x}, \widetilde{y}$. Пусть $\exists z: \; z \in \widetilde{x} \cap \widetilde{y}$. Тогда  $\left. \begin{array}{l} \widetilde{z} = \widetilde{x} \\ \widetilde{z} = \widetilde{y} \end{array} \right\} \Rightarrow \widetilde{x} = \widetilde{y}$
    \end{itemize}

\begin{definition}
    Мультимножество --- $(x; \varphi): \; \varphi \to \mathbb{Z}_+$ 
\end{definition}

Есть еще несколько базовых понятий: $k$-перестановки/сочетания из $n$ элементов с/без повторений.

$|A \cup B| = |A| + |B|$, если  $A \cap B = \varnothing$. Поэтому, если есть разбиение на блоки, то  $X=X_1\cup\ldots\cup X_k \Rightarrow |X| = |X_1|+\ldots+|X_k|$ 

$X = X_1 \times \ldots \times X_k$, тогда $|X| = |X_1| \cdot \ldots \cdot |X_k|$ 

$|A' \cap B'| = |(A \cup B)'| = |X| - |A \cup B| = |X| - |A| - |B| + |A \cap B|$
