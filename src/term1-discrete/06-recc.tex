\Subsection{Определение}
\begin{definition}
Пусть есть последовательность $(a_0, a_1, a_2,\ldots)$ и $a_{n+1} = F(a_0,a_1,\ldots)$. Тогда данная последовательность реккурентая.
\end{definition}
\slashn
Будем рассматривать последовательности, в которых $n$-ый считается от фиксированного количества предыдущих членов.
 \begin{example}
     Разводим лягушек. Изначально есть 50 лягушек. Каждый год количество увеличивается в 4 раза, но сто лягушек едут во Францию (навсегда\ldots). Тогда количество лягушек в $i$-ый год:  $a_n = 4a_{n-1} - 100$.
\end{example}
\slashn
Очень классно, но что с этим можно сделать? Все просто --- есть проблема в скорости пересчета, поэтому хочется найти замкнутую форму (формулу).

Но не для всех можно придумать формулу, конечно, не всегда. Но такие последовательности от дьявола.
\Subsection{Линейные реккуренты}
\begin{definition}
    Линейными реккурентным соотношениями будем называть реккуренты вида: \[
        a_{n+m} = b_1(n) \cdot a_{n+m-1} + b_2(n)a_{n+m-2} + \ldots + u(n)
    .\]Где $b_i(n) = \text{const} = u(n)$.
\end{definition}
\begin{definition}
    Соотношение однородное, если $u(n) = 0$.
\end{definition}
\slashn
Если соотношение однородное, то можно сказать $a_n = \lambda^n$! Что просто замечательно!
     $a_{n+2} = b_1 a_{n+1} + b_2 a_n$. Тогда $\lambda^{n+2} = b_1 \lambda^{n+1} + b_2 \lambda^{n}$.
     
     Тогда получаем, $\lambda^2 = b_1 \lambda + b_2$ --- \textbf{характеристическое уравнение реккурентного соотношения}. 

     Пусть мы в решении мы нашли два неравных решения $\lambda_1, \lambda_2$. Тогда заметим, что их сумма подходит. А еще домножение каждого на константу работает.

     То есть  $a_n = c_1 \lambda_1^n + c_2 \lambda_2^n$, $\forall c_1, c_2$. Тогда нам можно выбрать просто $a_0, a_1$.

     Заметим, что по $a_0, a_1$ можно найти $c_1, c_2$: $\begin{cases} a_0 = c_1 \lambda_1^0 + c_2 \lambda_2^0 =c_1 + c_2 
     \\ a_1 = c_1 \lambda_1 + c_2 \lambda_2\end{cases}$. Откуда получаем, что $c_2 = \frac{a_1-\lambda_1a_0}{\lambda_2 - \lambda_1}$ и $c_1 = a_0 - c_2$.

     Теперь разберем случай, когда $\lambda_1 = \lambda_2$. Тогда будем искать вид $a_n = c_1 \lambda_1^n + c_2 \cdot n \cdot \lambda_1^n$.
\begin{proof}
Хотим доказать:
    \[
 c_1 \cdot \lambda^{n+2} + c_2 (n+2)\lambda^{n+2} = b_1c_1\lambda^{n+1} + c_2 (n+1) \lambda^{n+1} + b_1c_1\lambda^{n} + c_2 (n) \lambda^{n}.
.\]
Заметим, что достаточно доказывать, что $c_1\ldots$ = $c_1\ldots$ и $c_2\ldots=c_2\ldots$. Тогда докажем, что штука $(n+2) \lambda_1^{n+2} = b_1(n+1)\lambda^{n+1} + b_2n\lambda^{n}$:
\[
    n \lambda_1^{n}+2\lambda^{n+2} = n\lambda_1^{n+1} + n\lambda_1^{n} + \lambda_1^{n+1}
.\] 
Заметим, что штуки с $n$ решается понятно как ($\lambda_1$ --- корень хар. уравнения). Тогда получили:
\[
    2\lambda_1^{n+2} = \lambda_1^{n+1} \iff 2\lambda_1^2 - \lambda_1 = 0
.\] 
Дальше решаем систему для $a_0, a_1$ и живем счастливо!.
\end{proof}

\begin{example}[Числа Фиббоначи]
\[F_0 = 0, F_1 = 1, F_{n+1} = F_n + F_{n-1}, F_n = \lambda^n.\]
\[\lambda^{n+1} = \lambda^{n} + \lambda^{n-1} \iff \lambda_{1,2} = \frac{1\pm\sqrt{5}}{2}\]
\[\begin{cases} F_0 = c_1(\frac{1+\sqrt{5}}{2})^0 + c_2 (\frac{1-\sqrt{5}}{2})^0 \\ F_0 = c_1(\frac{1+\sqrt{5}}{2})^1 + c_2 (\frac{1-\sqrt{5}}{2})^1\end{cases}.\]
Откуда получаем, что $c_1 = \frac{1}{\sqrt{5}}, c_2=-\frac{1}{\sqrt{5}}$. 
\end{example}
\slashn
Пусть у нас больше двух членов в реккуренте. Заметим, что там техника будет ровно такая же. Только теперь получим $a_n = \sum c_i \lambda_i^n$. Но пусть у лямбды есть кратность, тогда будем искать: $a_n = c_1\lambda_1^n + c_2 n \lambda_1^n + c_2 n^2 \lambda_1^n + \ldots$. Соответственно, если кратность $ds$, то для лямбды будет  $\sum c_i n^i \lambda^n$. 

\Subsection{Неоднородные линейные реккуренты}
Пусть есть $a_{n+1} =4a_n - 100$. Тогда скажем, что на самом деле  $a_{n+1} = b a_n + 4 = b(ba_{n-1} + 4) = b(b(ba_{n-2} + 4) + 4) + 4 = \ldots = b^{n+1}a_0 + (b^n + b^{n-1} + \ldots + b + 1) \cdot 4 = b^{n+1} \cdot a_0 + \frac{b^{n+1} - 1}{b - 1} \cdot 4$.

\begin{theorem}
    $a_{n+m} = b_1a_{n+m-1} + \ldots + b_m a_n + u(n)$. Если $\alpha_n$ --- решение левого, а  $\beta_n$ --- удовлетворяет тому же, но без $u(n)$. То  $\alpha_n + c \beta_n$ будет удовлетворять реккуренте. 
\end{theorem}
\begin{proof}
     \colorbox{pink}{$\alpha_{n+m}$} $+$ \colorbox{green}{$c \beta_{n+m}$} $=$ \colorbox{green}{$b_1$}$($\colorbox{pink}{$\alpha_{n+m-1}$} $+$ \colorbox{green}{$\beta_{n+m-1}$}$)$ $+$ \ldots $+ u(n).$
\end{proof}
\begin{example}
    $a_{n+1} = 2a_n + 7$.  $a_n = C$, тогда  $c = -7$.  $a_{n+1} = 2a_n \Rightarrow a_n = c 2^n$.  $a_0 = c \cdot 2^0 - 7 \Rightarrow C = a_0 + 7$
\end{example}
\begin{example}
    $a_{n+1} = 2a_n + (n+1)3^n$. Будем искать частное решение вида  $(b_1n+b_0)3^n$: \[
        b_1(n+1)3^{n+1}+b_{0} 3^{n+1} = 2 b_1 3^n + 2 b_1b_0 3^n + (n+1)3^n
    .\] 
    Сокращаем на $3^n$:  \[
    3b_1n + 3b_1 + 3b_0 = n + 2b_1 + b_0 + 1.
    .\] Что выполняется для любого $n$. Тогда  $3b_1 = 1$ и $3b_1 + 3b_0 = 2b_1 + 2b_0 + 1$
\end{example}
\begin{example}
    $a_{n+1} = a_n + 1$. Заметим, что здесь  $c = c+1$ уже не подходит.  А характеристическое уравнение:  $a_{n+1} = a_n \Rightarrow \lambda = 1$.
\end{example}
\begin{example}
    $a_{n+2} = 7a_{n+1} + 11a_n + 7^n + (n+1)3^n$. Последние два слагаемые нельзя представить в виде $P(n)R^n$. Тогда можно отдельно решить без них, с первым с двумя. А дальше как обычно.
\end{example}
