\Subsection{Дифференцируемость функции многих переменных}
\begin{definition}
    $f\!: E \subset \R^n \to R^m$,  $a \in \Int E$.

     $f$ --- дифференцируема в точке  $a$, если  $f(a+h) = f(a) + Th + o(\|h\|)$ при  $h \to \overrightarrow{0}$, где  $T\!: \R^n \to \R^m$ --- линейное отображение.
\end{definition}
\begin{remark}
    Если $f$ дифференцируема в точке  $a$, то  $T$ определена однозначно. 
\end{remark}
\begin{proof}
    $f(a+th) = f(a) + T(th) + o(\|th\|) = f(a) + tTh + o(t\|h\|) = f(a) + tTh + o(t)$. $\|h\|$ --- константа, поэтому можно выкинуть.

     $Th = \lim\limits_{t \to 0} \frac{f(a+th) - f(a)}{t}$
\end{proof}
\begin{remark}
    Если $f$ дифференцируем в точке  $a$, то  $f$ непрерывна в точке  $a$.

     $f(a+h) = f(a) + Th + o(\|h\|) \xrightarrow{h \to 0} f(a)$.
\end{remark}
\begin{definition}
    $T$ --- дифференциал $f$ в точке  $a$. Обозначается  $\mathrm{d}_af$. 
\end{definition}
\begin{definition}
    Матрицу отображения $\mathrm{d}_af$ назовем матрицей Якоби  $f$ в точке  $a$.
\end{definition}

Важный частный случай $m = 1$.  $f(a+h) = f(a) + \langle v, h \rangle o(\| h\|)$. $V$ --- градиент функции  $f$ в точке  $a$. Обозначается  $\text{grad}\ f$,  $\nabla f(a)$ (набла).

\begin{theorem}
Пусть $f = \begin{pmatrix} f_1 \\ f_2 \\ \vdots \\ f_n \end{pmatrix}\!: E \to \R^m$, $a \in \Int E$.

Тогда $f$ --- дифференцируема в точке  $a \iff f_j$ дифференцируема в точке  $a \forall j$.
\end{theorem} 
\begin{proof}
    $f(a+h) = f(a) + Th + \alpha(h)$, где  $\frac{\alpha(h)}{\|h\|} \xrightarrow{h \to 0} 0$.
    \begin{itemize}
        \item $\Rightarrow$.  $f_j(a+h) = f_j(a) + (Th)_j + \alpha_j(h)$. Надо доказать, что  $\frac{\alpha_j(h)}{\|h\|} \to 0$. $|\alpha_j(h)| \le \|\alpha(h)\|$. Значит, $\frac{|\alpha_j(h)|}{\|h\|} \le \frac{\|\alpha_j(h)\|}{\|h\|} \to 0$.
        \item $\Leftarrow$. Знаем, что  $f_j(a+h) = f_j(a) + T_jh + \alpha_j(h)$, где  $\frac{\alpha_j(h)}{\|h\|} \to 0$. 

            $\alpha(h) = \begin{pmatrix} \alpha_1(h) \\ \vdots \\ \alpha_m(h) \end{pmatrix}$. Надо доказать, что $\frac{\|\alpha(h)\|}{\|h\|} \to 0$.

            Заметим, что $\frac{\|\alpha(h)\|}{\|h\|} = \frac{\sqrt{\alpha_1(h)^2 + \ldots + \alpha_m(h)^2}}{\|h\|} \le  \frac{|\alpha_1(h)| + \ldots + |\alpha_m(h)|}{\|h\|} \to 0$.
    \end{itemize}
\end{proof}
\begin{consequence}
    Строки матрицы Якоби --- градиенты координатных функций
\end{consequence}
\begin{proof}
    $Th = \begin{pmatrix} T_1 \\ T_2 \\ \vdots \\ T_m \end{pmatrix}(h) = \begin{pmatrix} T_1h \\ \vdots \\ T_mh \end{pmatrix}$.
\end{proof}
\begin{definition}
    Производная по направлению. $f\!: E \to \R, \|h\|=1$.

     $\frac{\partial f}{\partial h}(a) \coloneqq \lim\limits_{t \to 0} \frac{f(a+th) - f(a)}{t}$.
\end{definition}
\begin{remark}
    $\frac{\partial f}{\partial h}(a) = \mathrm{d}_af(h)$.
\end{remark}
\begin{theorem}[экстремальное свойство градиента]
    $\left| \frac{\partial f}{\partial h}(a) \right| \le \|\nabla f(a)\|$, причем равенство достигается при $h = \pm \frac{\nabla f(a)}{\|\nabla f(a)\|}$.
\end{theorem}
\begin{proof}
    $\frac{\partial f}{\partial h}(a) = \mathrm{d}_af(h) = \langle \nabla f(a), h \rangle$.
\\
    $\left| \frac{\partial f}{\partial h}(a) \right| = | \langle \nabla f(a), h| \le \|\nabla f(a) \| \cdot \|h\| = \|\nabla f(a)\|$.

    В неравенстве Коши-Буняковского равенство $\iff$ векторы пропорциональны  $\implies h = \pm \frac{\nabla f(a)}{\|\nabla f(a)\|}$.
\end{proof}
\begin{definition}
    Частные производные. $f\!: E \to \R$ (это $\iff f$ --- скалярная), $e_k$ --- базисный вектор (везде нули кроме $k$-й позиции).

     $\frac{\partial f}{\partial x_k}, f'_{x_k}, D_{x_k}f,\ldots$. $\frac{\partial f}{\partial x_k} (a) \coloneqq \frac{\partial f}{\partial e_k}(a)$.
\end{definition}
\begin{remark}
    $f(x_1, x_2, \ldots, x_n)$. $\frac{\partial f}{\partial x_1}(a) = \lim\limits_{t \to 0} \frac{f(a_1 + t, a_2, \ldots,a_n) - f(a_1, a_2,\ldots, a_n)}{t} = g'(a_1)$, где $g(s) \coloneqq f(s, a_2, \ldots, a_n)$.
\end{remark}
\begin{example}
    $f(x, y) = x^y$.  $\frac{\partial f}{\partial x} = yx^{y-1}$, $\frac{\partial f}{\partial y} = x^y \cdot \ln x$.
\end{example}
\begin{theorem}
    $\nabla f(a) = \left( \frac{\partial f}{\partial x_1}(a), \frac{\partial f}{\partial x_2}(a), \ldots, \frac{\partial f}{\partial x_n}(a) \right)$.
\end{theorem}
\begin{proof}
   $\frac{\partial f}{\partial h} = \langle \nabla f(a), h \rangle$.

   $\frac{\partial f}{\partial x_k}(a) = \frac{\partial f}{\partial e_k}(a) = \langle \nabla f(a), e_k \rangle = (\nabla f(a))_k$.
\end{proof}
\begin{consequence}
    $f\!: E \subset \R^n \to \R^m$,  $a \in \Int E$,  $f$ --- дифференцируема в точке  $a$.

    Тогда  $f'(a) = \begin{pmatrix} \frac{\partial f_1}{x_1} & \frac{\partial f_1}{x_2} & \ldots & \frac{\partial f_1}{x_n}\\  \frac{\partial f_2}{x_1} &\frac{\partial f_2}{x_2} & \ldots &\frac{\partial f_2}{x_n}\\ \ldots & \ldots & \ldots & \ldots\\ \frac{\partial f_m}{x_1} & \frac{\partial f_m}{x_2} & \ldots & \frac{\partial f_m}{x_n} \end{pmatrix}$.
\end{consequence}
\begin{theorem}(линейность дифференциала)
    $f, g\!: E \subset \R^n \to \R^m$,  $a \in \Int E$,  $\lambda \in \R$, $f, g$ дифференцируемы в точке $a$.

    Тогда $f + g, \lambda f$ --- дифференцируемы в  $a$ и $\mathrm{d}_a(f+g) = \mathrm{d}_af + \mathrm{d}_ag $ и $\mathrm{d}_a(\lambda f) = \lambda \mathrm{d}_a f$.
\end{theorem}
\begin{proof}
    $f(a+h) = f(a) + \mathrm{d}_af(h) + \alpha(h)$, $\frac{\alpha(h)}{\|h\|} \to 0$, $g(a + h) = g(a) + \mathrm{d}_ag(h) + \beta(h)$, $\frac{\beta(h)}{\|h\|} \to 0$.

    Тогда $f(a+h) + g(a+h) = f(a) + g(a) + \mathrm{d}_af(h) + \mathrm{d}_ag(h) + \alpha(h) + \beta(h)$. Считаем, что  $\alpha(h) + \beta(h) = o(\|h\|)$.

     $\lambda f(a+h) = \lambda f(a) + \lambda \mathrm{d_a}f(h) + \lambda\alpha(h)$.
\end{proof}
\begin{theorem}[дифференцируемость композиции]
    $f\!: E \subset \R^n \to D \subset \R^m$, $g\!: D \subset \R^m \to \R^l$,  $a \in \Int E, b = f(a) \in \Int D$.

    Тогда, если  $f$ дифференцируема в  $a$,  $g$ дифференцируема в  $b = f(a)$, то  $g \circ f$ дифференцируема в точке  $a$ и  $\mathrm{d}_a(g \circ f) = \mathrm{d}_{f(a)}g \circ \mathrm{d}_af$.
\end{theorem}
\begin{proof}
    $f(a+h) = \underbracket{f(a)}_{=b} + \underbracket{\mathrm{d}_af(h) + \alpha(h)}_{=k}$, где  $\frac{\alpha(h)}{\|h\|} \xrightarrow{h \to 0} 0$.

    $g(b+k) = g(b) + \mathrm{d}_bg(k) + \beta(k)$, где  $\frac{\beta(k)}{\|k\|} \xrightarrow{k \to 0} 0$.

    $g \circ f(a + h) = g(b + k) = g(b) + \underbracket{\mathrm{d}_bg(\mathrm{d}_af(h)+\alpha(h))}_{\mathrm{d}_bg(\mathrm{d_af(h)}) + \mathrm{d}_bg(\alpha(h))} = g \circ f(a) + \mathrm{d}_bg \circ \mathrm{d}_af(h) + \mathrm{d}_bg(\alpha(h)) + \beta(k)$.

    Хотим показать, что все корректно.

    $\frac{\|\mathrm{d}_bg(\alpha(h))\|}{\|h\|} \le \|\mathrm{d}_b(g)\| \underbracket{\frac{\|\alpha(h)\|}{\|h\|}}_{\to 0}$. $k = \mathrm{d}_af(h) + \alpha(h)$.  $\|k\| \le \|\mathrm{d}_af(h)\| + \|\alpha(h)\| \le \|\mathrm{d}_af\| \cdot \|h\| + \|\alpha(h)\| \to 0$, так как $\frac{\|k\|}{\|h\|} \le \|\mathrm{d}_af\| + \frac{\|\alpha(h)\|}{\|h\|}$.

    В итоге, $\frac{\|\beta(k)\|}{\|k\|} \cdot \frac{\|k\|}{\|h\|} \xrightarrow{h \to 0} 0$.
\end{proof}
\begin{consequence}
    $(g \circ f)'(a) = g'(f(a)) f'(a)$.
\end{consequence}
\begin{theorem}[Дифферециал произведения скалярной и векторной функции]
    $E \subset \R^n, a \in \Int E$,  $\lambda\!: E \to \R, f\!: E \to \R^m$,  $\lambda$ и  $f$ дифференцируемы в точке  $a$. Тогда  $\lambda f$ дифференцируема в точке  $a$ и  $\mathrm{d}_a(\lambda f)(h) = \mathrm{d}_a \lambda(h) f(a) + \lambda(a) \cdot \mathrm{d}_af(h)$.
\end{theorem}
\begin{proof}
    $f(a+h) = f(a) + \mathrm{d}_af(h) + \alpha(h)$,  $\frac{\alpha(h)}{\|h\|} \to 0$. $\lambda(a + h) = \lambda(a) + \mathrm{d}_a\lambda(h) + \beta(h), \frac{\beta(h)}{\|h\|} \to 0$.

    $\lambda(a+h)f(a+h) = \lambda(a)f(a) + {\color{orange}\mathrm{d}_a\lambda(h)f(a) + \lambda(a)\mathrm{d}_af(h)} + {\color{purple}\mathrm{d}_a\lambda(h)\cdot \mathrm{d}_af(h)} + \lambda(a) \cdot \alpha(h) + f(a) \beta(h) + {\color{blue}\mathrm{d}_af(h)\beta(h)} + {\color{olive}\mathrm{d}_a\lambda(h)\alpha(h)} +\alpha(h)\beta(h)$. 

    Заметим, что {\color{orange}второе и третье} слагаемые очевидно подходят под наше свойство. Теперь заметим, что  {\color{purple}$\text{const} \cdot \|h\|^2 = o(\|h\|)$} и  {\color{blue}$\text{const} \cdot \|h\| \beta(h) = o(\|h\|)$} и  {\color{olive}$\text{const} \|h\| \|\alpha(h)\| = o(\|h\|)$}. И всё получается. 
\end{proof}
\begin{theorem}[о дифференциале скалярного произведения]
    $f, g\!: E \to \R^m, a \in \Int E$,  $f, g$ --- дифференцируемы в  $a$. 

    Тогда  $\langle f, g\rangle$ дифференцируемы в  $a$. $\mathrm{d}_a \langle f, g \rangle(h) = \langle \mathrm{d}_af(h), g(a) \rangle + \langle f(a), \mathrm{d}_ag(h) \rangle$.
\end{theorem}
\begin{proof}
    $\langle f(x), g(x) = \sum\limits_{j=1}^m f_j(x) g_j(x)$.

     $\mathrm{d}_a\langle f, g \rangle(h) = \sum\limits_{j=1}^m \mathrm{d}_a(f_jg_j)(h) = \sum\limits_{j=1}^m (\mathrm{d}_a g_j(h)f_j(a) + g_j(a) \mathrm{d}_af_j(h)) = \sum\limits_{j=1}^m f_j(a) \mathrm{d}_ag_j(h) + \sum\limits_{j=1}^m \mathrm{d}_a f_j(h)g_j(a) = \langle f(a), \mathrm{d}_ag(h)\rangle + \langle \mathrm{d}_a f(h), g(a) \rangle$.
\end{proof}
\begin{remark}
    При $n=1$  $\langle f, g \rangle'(a) = \langle f'(a), g(a)\rangle + \langle f(a), g'(a) \rangle$.
\end{remark}
\begin{theorem}[Лагранжа для векторнозначных функций]
    $f\!: [a, b] \to \R^m$,  $f$ --- дифференцируема во всех точках из  $a, b$ и непрерывна на  $[a, b]$.

    Тогда существует  $c \in (a, b)$, такая что  $\|f(b) - f(a)\| \le \|f'(c)\|(b-a)$.
\end{theorem}
\begin{proof}
    $\vphi(t) \coloneqq \langle f(t), f(b) - f(a) \rangle$ --- дифференцируемая функция  $\implies \exists c \in (a, b)\!: \|f(b) - f(a)\|^2 = \langle f(b), f(b) - f(a) \rangle - \langle f(a), f(b) - f(a) \rangle = \vphi(b) - \vphi(a)= \vphi'(c)(b-a) = (b-a) \langle f'(c), f(b) - f(a) \rangle \le (b-a) \|f'(c)\|\|f(b) - f(a)\|$.\\
    $\vphi'(t) = \langle f'(t), f(b) - f(a) \rangle$.
\end{proof}
\begin{example}
    $m = 2$,  $[a, b] = [0, 2\pi], f(t) = \begin{pmatrix} \cos t \\ \sin t \end{pmatrix}, f'(t) = \begin{pmatrix} -\sin t \\ \cos t \end{pmatrix}$. $f(2\pi) - f(0) = \overrightarrow{0}$,  $\|f'(t)\| = 1$. Тогда получаем  $\|f(2\pi) - f(0)\| = 0 < 2\pi\|f'(c)\|$.
\end{example}
\Subsection{Непрерывная дифференцируемость}
\begin{theorem}
    $f\!: E \subset \R^n \to \R$,  $a \in \Int E$. Все частные производные функции  $f$ существуют в окрестности  $a$ и непрерывны в ней. Тогда  $f$ дифференцируема в точке  $a$.
\end{theorem}
\begin{proof}
    $f(a+h) - f(a) = \sum\limits_{i=1}^n \frac{\partial f}{x_i}(a) h_i + o(\|h\|)$, $R(h) = f(a + h) - f(a) - \sum\limits_{i=1}^n \frac{\partial f}{\partial x_i}(a) h_i$.

    $b_0 = a, b_1 = (a_1 + h, a_2, \ldots, a_n), b_k = (a_1 + h_1, a_2 + h_2, \ldots, a_k + h_k, a_{k+1}, \ldots, a_n).$

    $f(b_k) - f(b_{k-1}) = f(a_1 + h_1, \ldots, a_{k-1} + h_{k-1}, a_k + h_k, a_{k+1}, \ldots, a_n) - f(a_1 + h_1, \ldots, a_{k-1} + h_{k-1}, a_k, \ldots, a_n) = h_k \frac{\partial f}{\partial x_k}(b_{k-1} - \Theta_k h_k e_k)$ ($0 < \Theta_k < 1$). 

    Тогда $f(a + h) - f(a) = f(b_n) - f(b_0) = \sum\limits_{k=1}^n h_k \frac{\partial f}{\partial x_k}(b_{k-1} + \Theta_k h_ke_k) = \sum\limits_{k=1}^n h_k \frac{\partial f}{\partial x_k}(a) + \sum\limits_{k=1}^nh_k(\frac{\partial f}{\partial x_k}(b_{k-1} + \Theta_k h_k e_k) - \frac{\partial f}{\partial x_k}(a))$.

    Тогда $|R(h)| = \left|\sum\limits_{k=1}^n h_k \left( \frac{\partial f}{\partial x_k}(b_{k-1} + \Theta_k h_k e_k) - \frac{\partial f}{\partial x_k}(a)\right)\right| \le \|h\| (\sum(...)^2)^{\frac{1}{2}}$, а $(\sum(...)^2)^{\frac{1}{2}} \to 0$.
\end{proof}
\begin{remark}
    Дифференциал функции в точке не влечет существования частных производных (и даже непрерывности) в окрестности точки. 

    Пример: $f(x, y) = \begin{cases} x^2 + y^2 & \text{если } x,y\in\Q\\ 0 & \text{иначе} \end{cases}$. Дифференциал в точке $(0, 0)$:  $f(x, y) = f(0, 0) + o(\sqrt{x^2 + y^2})$ при  $(x, y) \to (0, 0)$.
\end{remark}
\begin{definition}
$f\!: E \subset \R^n \to \R^m$  $f$ непрерывно дифференцируема в каждой точке и  $\dd_xf$ непрерывно зависит от точки, то есть $\|\dd_xf - \dd_af\| \to 0$  при $x \to a$.
\end{definition}
\begin{theorem}
    $f$ непрерывно дифференцируема  $\iff$  $f$ дифференцируема во всех точка и все частные производные непрерывны. 
\end{theorem}
\begin{remark}
    $A = \begin{pmatrix} a_{11} & \ldots & a_{1m}\\ \vdots & \ddots & \vdots \\ a_{n_1} & \ldots & a_{nm} \end{pmatrix}$, $a_{jk} = \langle Ae_k, e_j \rangle$,  $|a_{jk}| = |\langle A_{e_k}, e_j \rangle| \le \|Ae_k\| \underbrace{\|e_j\|}_{=1} \le \|A\| \cdot \|e_k\| = \|A\|$.
\end{remark}

\begin{proof}
    \begin{itemize}
        \item $\Rightarrow$. $\|\dd_xf - \dd_af\| \to 0$. \\ Матрица для  $\dd_xf - \dd_af = \begin{pmatrix} \frac{\partial f_1}{\partial x_1}(x) - \frac{\partial f_1}{\partial x_1}(a) & \ldots & \frac{\partial f_1}{\partial x_n}(x) - \frac{\partial f_1}{\partial x_n}(a) \\ \vdots & \ddots & \vdots \\ \frac{\partial f_m}{\partial x_1}(x) - \frac{\partial f_m}{\partial x_1}(a) & \ldots & \frac{\partial f_m}{\partial x_n}(x) - \frac{\partial f_m}{\partial x_n}(a) \end{pmatrix}$. $\left|\frac{\partial f_j}{\partial x_k}(x) - \frac{\partial f_j}{\partial x_k}(a)\right| \le \|\dd_xf-\dd_af\| \xrightarrow{x\to a} 0 \implies \frac{\partial f_j}{\partial x_k}$ непрерывна в $a$.
        \item $\|\dd_xf - \dd_af\|^2 \le \sum\limits_{j=1}^m \sum\limits_{k=1}^n \left( \frac{\partial f_j}{\partial x_k}(x) - \frac{\partial f_j}{\partial x_k}(a) \right)^2 \to 0$. 
   \end{itemize} 
\end{proof}
\begin{consequence}
   $f$ --- непрерывно дифференцируема  $\iff$ существуют все частные производные и они непрерывны.
\end{consequence}
\begin{theorem}
    Линейная комбинация, композиция, скалярное произведение, непрерывной дифференцируемой функции --- непрерывно дифференцируема.
\end{theorem}
\Subsection{Частные производные высших порядков}
\begin{definition}
    $f\!: E \subset \R^n \to \R, f(x_1, x_2, \ldots, x_n)$.

    $\frac{\partial f}{\partial x_j}\!: E \to \R$. Тогда считаем, что $\frac{\partial^2 f}{\partial x_i \partial x_j}(x) = \frac{\partial}{\partial x_i}\left(\frac{\partial f}{\partial f)j}\right)$
\end{definition}
\begin{definition}[Обозначения]
    $f''_{x_j, x_i}$ или  $D_{x_i, x_j}f$.    

    $\frac{\partial^r f}{\partial x_{i_r} \partial x_{i_{r-1}} \ldots \partial x_{i_1}} \coloneqq \frac{\partial}{\partial x_{i_r}} \left( \frac{\partial^{r-1}}{\partial x_{i_{r-1}} \ldots x_{i_1}}\right)$
\end{definition}
\begin{remark}
    Всего $n^r$ частных производных порядка $r$.
\end{remark}
\begin{example}
    $f(x, y) = x^y$,  $\frac{\partial f}{\partial x} = y x^{y-1}$, $\frac{\partial f}{\partial y} = x^y \ln x$.

    $\frac{\partial^2 f}{\partial x^2} = y(y-1)x^{y-2}$

    $\frac{\partial^2 f}{\partial y^2} = x^y (\ln x)^2$.

    $\frac{\partial^2 f}{\partial y \partial x} = x^{y-1} + y x^{y-1} \ln x$.

    $\frac{\partial^2 f}{\partial x \partial y} = x^{y-1} + y x^{y-1} \ln x$.
\end{example}
\begin{example}
    $f(x, y) = \begin{cases} xy \frac{x^2 - y^2}{x^2 + y^2} & \text{если } x^2 + y^2 \neq 0\\ 0 & \text{иначе} \end{cases}$.

    $\frac{\partial f}{\partial x}(x, y) = \left( \frac{x^2y - xy^2}{x^2+y^2}\right)'_X = \frac{(3x^2y-y^3)(x^2+y^2) - 2x(x^3y-xy^3)}{(x^2+y^2)^2} = \frac{x^4y-y^5+4x^2y^3}{(x^2+y^2)^2}$.

    $\frac{\partial f}{\partial x}(0, 0) = \lim\limits_{x \to 0} \frac{f(x, 0) - f(0, 0)}{x} = \lim\limits_{x \to 0} \frac{0 - 0}{x} = 0$.

    $\frac{\partial^2 f}{\partial y \partial x}(0, 0) = \lim\limits_{y \to 0} \frac{\frac{\partial f}{\partial x}(0, y - \frac{\partial f}{\partial x}(0, 0))}{u} = \lim\limits_{y \to 0}\frac{-y - 0}{y} = -1$. А если в другом порядке, то 1.
\end{example}
\begin{theorem}
    $f\!: E \subset \R^2 \to \R, (x_0, y_0) \in \Int E$.

    $f'_x, f'_y, f''_{xy}$ существуют в окрестности  точки  $(x_0, y_0)$ и $f_{xy}''$ непрерывны в точке  $(x_0, y_0)$.

    Тогда существует $f''_{yx}(x_0, y_0)$ и $f''_{xy}(x_0, y_0) = f''_{yx}(x_0, y_0)$.
\end{theorem}
\begin{proof}
    $\vphi(s) \coloneqq f(s, y_0 + k) - f(s, y_0)$, $\vphi'(s) = f'_x(s, y_0 + k) - f'_x(s, y_0)$.

    $\vphi(x_0+h) - \vphi(x_0) = h'\vphi(x_0 + \Theta h) = h(f'_x(x_0 + \Theta h, y_0 + k) - f'_x(x_0 + \Theta h, y_0)) = hk f''_{xy}(x_0 + \Theta h, y_0 + \widetilde{\Theta}k)$.

    $\frac{\vphi(x_0 + h) - \vphi(x_0)}{hk} - f''_{xy}(x_0, y_0) = f''_{xy}(x_0 + \Theta h, y_0 + \widetilde{\Theta}k) - f''_{xy}(x_0, y_0) \to 0$, при $(h, k) \to (0, 0)$.

    При малых $h$ и  $k$  $\left| \frac{1}{h} \left(\frac{\vphi(x_0 + h)}{k} - \frac{\vphi(x_0)}{k}\right) - f''_{xy}(x_0, y_0) \right| < \eps \xrightarrow{k \to 0} \left| \frac{1}{h} \left( f'_y(x_0+h, y_0) - f'_y(x_0, y_0) \right) - f''_{xy}(x_0, y_0) \right| \le 0 \implies \lim\limits_{h \to 0} \frac{f'_y(x_0+h, y_0) - f'_y(x_0, y_0)}{h} = f''_{xy}(x_0, y_0)$.
\end{proof}
\begin{definition}
    $f\!: D \subset \R^n \to \R$ ($D$ --- открытое), $f$  $r$ раз непрерывно дифференцируема, если у нее существуют все частные производные порядка  $\le v$ и они непрерывны.

    Обозначение. $f \in C^r(D)$.
\end{definition}
\begin{theorem}
    $f \in C^r(D)$ и $(i_1, i_2, \ldots, i_r)$ перестановка $(j_1, j_2,\ldots,j_r)$.

Тогда $\frac{\partial^r f}{\partial x_{i_r}\ldots\partial x_{i_1}} = \frac{\partial^r f}{\partial x_{j_r} \ldots \partial x_{j_1}}$.
\end{theorem}
\begin{proof}
    Достаточно доказать данное утверждение для одной транспозиции ($a, b \to b, a$), тогда заметим, что у на есть общий префикс + общий суффикс + пара различных переменных. 
\end{proof}
\begin{definition}
    Мультииндекс $k = (k_1, k_2, \ldots, k_n)$ --- набор из $n$ неотрицательных чисел.

    $k! \coloneqq k_1!k_2!\ldots k_n!$, $|k| \coloneqq k_1 + k_2 + \ldots + k_n$ --- высота мультииндекса.

    $\binom{k_1 + k_2 + \ldots + k_n}{k_1, k_2, \ldots, k_n} \frac{|k|!}{k!}$.

    $h=(h_1, h_2, \ldots, h_n), h^k \coloneqq h_1^{k_1}\ldots h_n^{k_n}$.

    $f^{(k)} = \frac{\partial^{|k|} f}{\partial x_1^{k_1} \ldots \partial x_n^{k_n}}$.
\end{definition}

\begin{lemma}
    $f \in C^r(D), D \subset \R^n, D$ --- открытое,  $[x, x + h] \subset D, F(t) \coloneqq f(x+th), t \in [0; 1]$.

    Тогда $F \in C^r[0, 1]$ и  $F^{(t)}(t) = \sum\limits_{|k| = l} \frac{l!}{k!}f^{(k)}(x + th)h^k$.
\end{lemma}
\begin{proof}
    $g \in C^1(D), G(t) = g(x+th)$.  $G'(t) = \sum\limits_{j=1}^n g'_{x_j}(x+th) h_j$.  $v(t) \coloneqq x+th$.

    Тогда  $G'(t) = g'(v(t))v'(t) = g'(x+th)h= \begin{pmatrix} g'_{x_1}(x+th) & \ldots & g'_{x_n}(x+th) \end{pmatrix} \begin{pmatrix} h_1 \\ h_2 \\ \vdots \\ h_n \end{pmatrix} = \sum\limits_{j=1}^n g'_{x_j}(x+th)h_j$.

    А значит $F^{(l)}(t) = \sum\limits_{j_l=1}^n \sum\limits_{j_{l-1}=1}^n \ldots \sum\limits_{j=1}^n f^{(l)}_{x_{j_1}, x_{j_2}, \ldots, x_{j_l}}(x+th)h_{j_1}h_{j_2}\ldots h_{j_l} = \sum\limits{|k| = l} \frac{l!}{k!} f^{(k)}(x+th)h^k$. Просто $f^{(l)}_{x_{j_1}, x_{j_2}, \ldots, x_{j_l}}$ $k_1$ раз $x_1$, $k_2$ раз $x_2$, \ldots
\end{proof}
\begin{theorem}[Многомерная формула Тейлора (с остатком в форме Лагранжа)]
    $f \in C^{n+1}(D), D \subset \R^n$,  $D$ --- открытое,  $f\!: D \to \R, [x, x+h] \subset D$. 

    Тогда существует  $\Theta \in (0, 1)$, что  $f(x+h) = \sum\limits_{|k| \le r} \frac{f^{(k)}(x)}{k!}h^k + \sum\limits_{|k| = r+1} \frac{f^{(k)}(x + \Theta h)}{k!} h^k$.
\end{theorem}
\begin{proof}
    $F(t) \coloneqq f(x+th)$,  $F(1) = \underbrace{\sum\limits_{l=0}^r \frac{F^{(1)}(0)}{l!}1^l}_{=\sum\limits_{|k|=l} \frac{f^{(k)}(x)}{k!}h^k} + \underbrace{\frac{f^{(k+1)}(\Theta)}{(k+1)!}1^{r+1}}_{\sum\limits_{|k|=r+1} \frac{f^{(k)}(x+\Theta h)}{k!}h^k}$
\end{proof}
\begin{remark}
\begin{enumerate}
    \item Многочлен Тейлора степени $r$.  $\sum\limits_{|k| \le r} \frac{f^{(k)}(a)}{k!}(x-a)^k$.
    \item $r=0$. Многомерная формула Лагранжа. 

        $f(x+h) = f(x) + \sum\limits_{j=1}^n f'_{x_j}(x+\Theta h)h_j = f(x) \langle \nabla f(x + \Theta h), h \rangle$.
    \item $n=2$ в координатной записи.

        $f(x, y) = f(a, b) + f'_x(a, b)(x-a) + f'_y(a, b)(y-b)  + \frac{1}{2}f''_{xx}(a, b)(x-a)^2 + \frac{1}{2} f''_{yy}(a, b)(y-b)^2 + f''_{xy}(a, b)(x-a)(y-b) + \ldots$.
\end{enumerate}    
\end{remark}
\begin{theorem}[с остатком в форме Пеано]
    $f \in C^r(D), a \in D \subset \R^n$,  $D$ --- открытое. Тогда при  $x \to a$

    $f(x) = \sum\limits_{|k| \le r} \frac{f^{(k)}(a)}{k!}(x-a)^k + o(\|x-a\|^r)$.
\end{theorem}
\begin{proof}
    Пишем формулу с остатком в форме Лагранжа для $r-1$.

    $f(x) = \sum\limits_{|k| \le r-1} \frac{f^{(k)}(x)}{k!}(x-a)^k + \sum\limits_{|k| = r} \frac{f^{(k)}(a + \Theta(x-a))}{k!}(x-a)^k = \sum\limits_{|k| \le r} \frac{f^{(k)}(a)}{k!}(x-a)^k + \underbracket{\sum\limits_{|k| = r} \frac{f^{(k)}(a + \Theta(x-a)) - f^{(k)}(a)}{k!}(x-a)^k}_{=o(\|x-a\|^r)}$.

    Достаточно проверить, что $\underbrace{(f^{(k)}(a+\Theta(x-a)) -f^{(k)}(a))}_{\xrightarrow{x \to a} 0}(x-a)^k= o(\|x-a\|^r)$.

    Надо понять, что  $|(x-a)^k| \le \|x-a\|^r$. Заметим, что $|x_j - a_j| \le \|x-a\|$, тогда можно воспользоваться определением.
\end{proof}
\begin{consequence}[Полиномиальная (мультиномиальная) формула]
    $(x_1 + x_2 + \ldots + x_n)^r = \sum_{|k| = r} \frac{r!}{k!}x^k = \sum_{|k| = r} \binom{r}{k_1, k_2, \ldots, k_n} x_1^{k_1}\ldots x_n^{k_n}$.
\end{consequence}
\begin{proof}
    $f(x) = (x_1 + x_2 + \ldots + x_n)^r = g^r(x)$, где $g(x) = x_1 + \ldots + x_n$.

    $f'_{x_j}(x) = rg^{r-1}(x) g'_{x_j}(x) = r g^{r-1}(x)$. Знаем, что если $|k| = r$, то  $f^{(k)}(x) = r!$. А если  $<$, то  $f^{(k)}(x) = \ldots \cdot g^{r-|k|}(x), f^{(k)}(0) = 0$.
\end{proof}
