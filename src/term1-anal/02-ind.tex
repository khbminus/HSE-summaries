\Subsection{Мат. индукции}
Пусть $P_n$ - последовательность утверждений. Тогда, если  $P_1$ --- верное и из того, что $P_n$ --- верно следует, что  $P_{n+1}$ --- верно. Тогда все $P_n$ верны  $\forall n \in \N$

\begin{definition}
    Пусть $A \subset \R$. Тогда  $A$ --- ограничено сверху, если  $\exists c \in \R: \; \forall a \in A\; a \le c$. Такое  $c$ называется верхней границей.
\end{definition}
\begin{definition}
    Пусть $A \subset \R$. Тогда  $A$ --- ограничено снизу, если  $\exists b \in \R: \; \forall a \in A\; a \ge b$. Такое  $b$ называется нижней границей.
\end{definition}
\begin{definition}
    Пусть $A \subset \R$. Тогда  $A$ --- ограничено, если оно ограничено сверху и снизу.
\end{definition}

\begin{example}
    $\N$ не ограничено сверху, но ограничено снизу.
\end{example}
\begin{proof}
    Пусть $\exists c \in \R: \; c \ge n\; \forall n \in \N$. Тогда это противоречит принципу Архимеда при $x = c, y = 1$. \\ Для ограниченности снизу достаточно взять $c=-1$.
\end{proof}
\Subsection{Наибольшие/наименьшие элементы}
\begin{theorem}
    В непустом конечном множестве $A$ есть наибольший и наименьший элементы.
\end{theorem}
\begin{proof}
    Докажем по индукции:
    \begin{itemize}
        \item База. $|A| = 1$. Очевидно.
        \item Переход.  $n \to n + 1$.
        \item Доказательство. Рассмотрим множество из  $n + 1$ элемента  $\{x_1\ldots x_n,x_{n+1}\}$. Выкинем из него последний элемент. Тогда по индукционному предположению у нас есть максимальный элемент $x_k$. Тогда рассмотрим два случая:
             \begin{enumerate}
                 \item $x_k \ge x_{n+1}$. Тогда $x_k$ --- наибольший элемент множества $\{x_1\ldots x_n,x_{n+1}\}$.
                 \item $x_k < x_{n+1}$. Тогда по транзитивности  $x_{n+1}$ больше всех других элементов множества. Значит, $x_{n+1}$ --- наибольший элемент множества $\{x_1\ldots x_n,x_{n+1}\}$.
            \end{enumerate}
    \end{itemize}
\end{proof}
\begin{theorem}
    В непустом ограниченном сверху (снизу) множестве целых чисел есть наибольший (наименьший) элемент.
\end{theorem}
\begin{proof}
    Пусть $A \subset \Z$.  $c$ --- его верхняя граница. 

    Возьмем  $b \in A$ и рассмотрим  $B \coloneqq {x \in A \mid x \ge b}$. Заметим, что $B$ содержит конечное число элементов, значит в нем есть наибольший элемент. Пусть это $m \in B$:  $\forall x \in B:\; x \le m$. Докажем, что $m$ --- наибольший элемент и в  $A$. 

    Для этого заметим, что любой $x \in A$ либо лежит в $B$, либо  $x < b$, а по транзитивности  $x < b \le m$.
\end{proof}

\begin{definition}
    Пусть $x \in \R$, тогда  $[x] = \lfloor x \rfloor$ --- наименьшее целое число, не превосходящее  $x$.
\end{definition}
\begin{enumerate}
    \item $[x]  \le x < [x] + 1$

        Левое неравенство очевидно. Правое неравенство можно доказать от противного: пусть $x \ge [x] + 1$, тогда справа целое число большое $[x]$, но меньшее  $x$. Противоречие.
    \item $x - 1 < [x] \le x$
\end{enumerate}

\begin{theorem}
    Если $x < y$ ($x, y \in \R$), то
    \begin{enumerate}
        \item $\exists r \in \Q: \; x < r < y$.
        \item $\exists r \notin \Q: \; x < r < y$
    \end{enumerate}
\end{theorem}
\begin{proof}[Пункт 1.]
    $\epsilon \coloneqq y - x > 0$.
    
    Найдется  $n \in \N: \; \frac{1}{n} < \eps = y - x$. Тогда $m \coloneqq [xn]+1$:  $r = \frac{m}{n}$ подходит.
    
    $\frac{m}{n} > x \iff [xn] + 1 = m > xn$ --- свойство целой части.
    $\frac{m}{n} < y$. $\frac{m-1}{n} = \frac{[nx]}{n} \le \frac{nx}{n} = x \Rightarrow \frac{m}{n} \le x + \frac{1}{n} < x + \epsilon = x + y - x = y$
\end{proof}
\begin{proof}[Пункт 2.]
    $\sqrt{2} \notin \Q$. Рассмотрим $x - \sqrt{2}$ <  $y - \sqrt{2} \Rightarrow \exists r \in Q: \; x - \sqrt{2} < r < y - \sqrt{2} \Rightarrow x < \underbrace{r + \sqrt{2}}_{r'} < y$. Почему  $r'$ иррационально? Иначе  $\sqrt{2} = r' - r \in \Q$.
\end{proof}
