 \begin{definition}
   $f: \; \N \to \R$ 
\end{definition}.
\slashn
Способы задания последовательностей
\begin{enumerate}
    \item Формулой. $f_n \coloneqq \frac{\sin n}{n^n}$ 
    \item Рекуррентой: $f_1 = 1, f_2=2, f_{n+2} = f_n + f_{n+1}$.
\end{enumerate}
Способы визуализации:
\begin{enumerate}
    \item Можно ставить точки на прямой. Но если последовательность, например, $a_n \coloneqq \sin(\frac{n \pi}{2})$, то получится кукож.
    \item График. Считаем значения в натуральных точках.
\end{enumerate}
\begin{definition}
    Последовательность $a_n$ ограничена сверху, если  $\exists C: \; \forall n \in \N: \; a_n \le c$.
\end{definition}
\begin{definition}
    Последовательность $a_n$ ограничена снизу, если  $\exists C: \; \forall n \in \N: \; a_n \ge c$.
\end{definition}
\begin{definition}
    Последовательность $a_n$ ограничена, если она ограничена и сверху, и снизу.
\end{definition}
\begin{definition}
    Последовательность $a_n$ монотонно возрастает, если  $a_1 \le a_2 \le a_3 \le \ldots$.
\end{definition}
\begin{definition}
    Последовательность $a_n$ строго монотонно возрастает, если  $a_1 < a_2 < \ldots$.
\end{definition}
\begin{definition}
    Последовательность $a_n$ монотонно убывает, если  $a_1 \ge a_2 \ge a_3 \ge \ldots$.
\end{definition}
\begin{definition}
    Последовательность $a_n$ строго монотонно убывает, если  $a_1 > a_2 > a_3 > \ldots$.
\end{definition}
\begin{definition}[Нетрадиционное определение предела]
    $l = \lim a_n \iff$ вне любого интервала, содержащего $l$ находится конечное число членов последовательности. 
\end{definition}
\begin{remark}
    Мы можем смотреть только на симметричные относительно точки $l$ интервалы. Если он не симметричен, то можно большую границу уменьшить. Так можно сделать, так как мы знаем, что вне меньшего конечное число точек, то и снаружи большего точно конечное число точек. Тогда наш интервал выглядит как $(l - \epsilon; l + \epsilon)$
\end{remark}
\begin{remark}
    Конечное число точек снаружи интервала $\iff$ начиная с некоторого номера все попали в интервал, так как возьмем последнюю точку вне интервалов, и взяли её номер + 1.
\end{remark}
\begin{definition}[Традиционное определение предела]
    $l = \lim a_n \iff \forall \epsilon > 0: \; \exists N: \; \forall n\ge N: \;  |a_n-l| < \epsilon$
\end{definition}
\begin{enumerate}
    \item Предел единственный. Пусть $l$ и  $l'$ единственный. \emph{(Картинка)}. Рассмотрим интервал содержащий  $l$, но не  $l'$. Снаружи конечное число точек, теперь наоборот, там тоже конечное число точек. Тогда последовательность конечна.
    \item Если из последовательности выкинуть какое-то число членов, то предел не изменится. Доказательство через картинку.
    \item Если как-то переставить члены последовательности, то предел не изменится. Ну очевидно, что количество членов не изменилось, точки не поменяли своё местоположение.
    \item Если члены последовательности записать с какой-то кратностью (конечной), то предел не изменится.
\end{enumerate}
\begin{example}
    $\lim \frac{1}{n} = 0$. Мы знаем, что найдется такой номер, что $\frac{1}{n} < \beta$, тогда при $n \ge N$ $0 < \frac{1}{n} \le \frac{1}{N} < \beta$
\end{example}
